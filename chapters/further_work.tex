
\section{Health Metrics Derived from the Workload Model}

In addition to quantifying the teaching workload of the faculty, the workload model can also be used to derive important insights regarding the overall health and sustainability of the institution. Certain indicators like the fact that the faculty are overloaded with work, or that the faculty are not making significant research contributions, can be derived during the workload modelling process. These indicators can then be used to make important decisions regarding the allocation of resources to the various departments, and the hiring of new faculty.

Some of these indicators are:

\begin{enumerate}
  \item \textbf{Disproportionate Distribution of Workload}

        Even though we imposed a maximum cap of 8 units on the research, and service workload of the faculty, reaching the maximum cap is not necessarily a good thing.

        If the maximum cap for service workload is reached, it means that the faculty is performing a disproportionately large amount of service duties. In certain circumstances, the dean of the school for example, redistribution of the service duty between two faculty might not be possible. In such cases, special exceptions need to be made to reduce the teaching and research workload below the thresholds that the RTS model allows for. In typical cases however, redistribution of the service duties between the faculty is possible, and should be done to reduce the service workload of the faculty.

        Similarly, the research workload ceiling being reached means the faculty is handling a disproportionately large number of research staff. This is not necessarily a bad thing, as the faculty might be in a research area that is currently popular, and thus have a large number of research staff working under them. However, to maintain optimal research output, the workload should be redistributed between the faculty to ensure that the faculty are not overloaded with work.

  \item \textbf{Faculty Overload}

        In certain circumstances, the total workload limit of 12 may be exceeded. For example, if a faculty is supervising 4 post-doctoral fellows, while also handling multiple service duties, the RTS ratio of the faculty might approach $R:T:S = 8:2:8$, exceeding the maxima by 6 units. This is an indication that the faculty is overloaded with work, and the workload should be redistributed between the faculty to ensure that the faculty are not overloaded with work.

  \item \textbf{Meeting Institution-wide Objectives}

        The workload model can also be used to ensure that the institutional objectives are being appropriately achieved. To ensure this, the institution has an average RTS ratio that may be aimed towards. A research-intensive university may aim for an aggregate RTS Ratio of $R:T:S = 6:4:2$, and if the aggregate RTS Ratio is leaning towards $R:T:S = 4:6:2$, course-correction is required, by refocusing the efforts towards research, or by hiring additional research-intensive staff to maintain the direction of the institution.
\end{enumerate}
