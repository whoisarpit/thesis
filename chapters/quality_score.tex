
\chapter{Quantifying Faculty Course Fit}

\section{Introduction}

Deciding which faculty member should teach a course is a difficult task. There are three main stakeholders in this decision: the students, the management, and the faculty. Each of these stakeholders has their own priorities and preferences. Their priorities can be conflicting and achieving the priorities of one stakeholder may compromise the priorities of another. Thus, an optimal solution is one that balances the priorities of all three stakeholders fairly. In this chapter, we will discuss the priorities of each stakeholder and how we can quantify them. We will then discuss how we can collect the preferences of each stakeholder and how we can define a unified course fit metric that balances the priorities of all stakeholders.


\section{Defining Characteristics of a Good Course Fit Metric}

The course fit metric should satisfy the following requirements:

\begin{itemize}

  \item {Balances the priorities of all stakeholders.}
  \item {Easy to understand and explain.}
  \item {Easy to collect the data required to compute it.}
  \item {Easy to tweak and improve.}
  \item {Easy to implement and compute.}

\end{itemize}

The first step towards defining a good course fit metric is to understand the priorities of each stakeholder. We will discuss the priorities of each stakeholder in the following sections.


\section{Student Priorities}

The students are the most important stakeholders in the decision of which faculty member should teach a course, since their learning experience and outcomes are directly affected by this decision. The students' priorities are as follows:

\begin{enumerate}
  \item {The faculty is knowledgeable about the subject matter.}

        One of the most important factors that affects the students' learning experience is the faculty's knowledge of the subject matter. If the faculty is not knowledgeable about the subject matter, they will not be able to answer the students' questions and they will not be able to teach the students the subject matter effectively. Thus, the students' learning experience will be negatively affected.

  \item {The faculty is effective at teaching.}

        Different faculty members have different capabilities and skill sets. While some faculty are great at research and writing papers, others are great at teaching. It is in the students' best interest to have a faculty member who is effective at teaching, since this will improve their learning experience and outcomes.

  \item {The faculty has a teaching style that is compatible with the students' learning style.}

        Students at different stages of their education have different learning styles. Freshmen and sophomores are new to the subject matter and they need a faculty member who can provide them with a lot of guidance and support to build a strong foundation. Juniors and seniors are more inclined towards self-learning and they need a faculty member who are subject-matter experts and can provide them with the resources they need to learn the subject matter on their own.

\end{enumerate}

A simple way to quantify the students' priorities is to collect their feedback about the faculty member who taught the course in the past. In the next section, we will discuss how we will be collecting this feedback.

\section{Collecting Student Feedback}

% TODO





\subsection{Management Priorities}

\subsection{Faculty Preferences}

\section{Collecting Faculty Preferences}


\section{Defining a Unified Course Fit Metric}
