\chapter{Conclusion and Future Work}

\section{Conclusion}

In this thesis, the problem of automated teaching workload allocation to faculty members was addressed. The proposed solutions were found to be holistic and effective, and were able to allocate all teaching workload to the faculty, while ensuring that the allocation was fair and equitable, and prioritized the student feedback, the faculty preferences, and the management priorities.

To allocate the courses, the problem was split into four sub-problems - the quantification of the workload involved in the teaching activities of a course, determining how much workload each faculty member should be assigned, allocating lectures to the faculty members, and allocating tutorials and lab sessions to the faculty members.

To quantify the workload involved in the teaching activities of a course, the activity was divided into four distinct stages, namely - preparation, review, delivery, and grading Workload. The preparation workload and grading workload were only applicable to the course lecturer, while the review workload and delivery workload applied to all faculty members involved in the teaching of the course. For each of these stages, the factors affecting the workload were identified. The factors were the course newness, class size, and activity type (lecture, tutorial, or lab). Using these factors, a model was developed to quantify the workload involved in the three teaching activities of the course in the form of Workload Units. The model was found to be effective and was able to quantify the workload of the teaching activities accurately and comprehensively in a variety of scenarios.

The workload of tutorials and lab sessions was found to be typically 20-40 units. It was found that the workload of a lecture can range anywhere from 80 to 600 units, being higher for larger class sizes, and if the course is being taught for the first time. The high workload of a lecture would result in difficulties finding eligible faculty to teach the lecture without overloading them. To alleviate this, a lecture splitting algorithm was developed, that splits the workload of a lecture into 2-3 smaller lectures, which can be allocated to multiple faculty members. The lecture-splitting algorithm was found to be effective in reducing the workload of high-workload lectures and was able to reduce the overloading of faculty.

To determine how much workload each faculty member should be assigned, the research and service workload of the faculty were quantified. The research workload was quantified in terms of the number of research staff that the faculty is supervising, while the service workload was quantified in terms of the service duties that the faculty is performing.

The workload was then modeled into an RTS ratio, to represent the Research, Service, and Teaching workload, the sum of the three components always being 12 since the total workload of each faculty is comparable. The RTS ratio was found to be a good representation of the workload of the faculty. The RTS ratio was then used to determine the $T$ ratio, by subtracting the research and service workload from the total workload of the faculty i..e 12. This $T$ ratio is the teaching workload that should be allocated to the faculty in relative terms.

To allocate lectures to the faculty members, the Hungarian Algorithm was used, which allocates one lecture per faculty member in every iteration. The $T$ ratio of the faculty, combined with the workload of the lecture, was used to determine the workload limit for the faculty, which was used to ensure that the faculty members were not overloaded with work.

It was found that lectures remained unallocated in certain cases, due to various reasons. In one case, \textbf{44\%} of the lectures remained unallocated. To handle this, various techniques were developed. These techniques were pre-allocating lectures with supply shortages, dynamic-splitting unallocated lectures into two smaller lectures to allocate to faculty, workload-limit-relaxation to allow for overloading of faculty, and Dynamic Swapping to free up some faculty who can teach the remaining unallocated lectures. As a result of these techniques, up to \textbf{98\%} of the lectures were allocated to the faculty. As a final step, an equal misery approach was devised, which allowed up to \textbf{99\% of the lectures to be allocated} to the faculty.

Similar techniques were used for the allocation of tutorials, with some changes to allocate one tutorial session per faculty in every iteration to allow the same faculty to teach multiple sessions of the same course tutorial. However, it was found that on average, a course had \textbf{4.5 different faculty teaching the tutorials of the course}. To avoid reducing this fragmentation, batching of tutorials into groups of up to 6 tutorials of the same course, and adding a consistency bias were introduced. These measures were found to be very effective and reduced the number of faculty teaching the tutorials of the same course to \textbf{1.2 faculty per course}.

Additionally, pre-allocation of tutorials was also implemented, similar to lecture pre-allocation, which was able to improve tutorial allocation by an additional 3\%, from 73\% to 76\%. Workload relaxation was avoided, in favor of a dynamic limit adjustment approach which is more targeted. The dynamic limit adjustment approach was found to be very effective and was able to \textbf{improve tutorial allocation from 76\% to 100\%}, allocating all tutorials to the faculty, without adversely affecting the workload limits of the faculty.

The allocation of labs was identical to tutorial allocation and with the combination of pre-allocation, dynamic limit adjustment, consistency bias, and all the other techniques, \textbf{100\% of the lab sessions were allocated to the faculty, with an average of 1.4 faculty teaching the labs of the same course}.

With the allocation of all teaching activities to the faculty using the Hungarian Algorithm, incorporating the RTS ratio and measuring the workload impact of the teaching activities, an allocation was achieved that was fair and equitable while prioritizing the student feedback, the faculty preferences, and the management priorities.

\section{Future Work}

The allocation of teaching workload to faculty is a complex problem, and while the allocation achieved in this thesis is fair and equitable, there are still certain areas of improvement that can be made. Some of these areas are discussed below.

\subsection{Workload Analysis from the RTS Model}

The RTS Model provides a lot of insight into the workload of the faculty and can be used to derive important metrics regarding the health of the institution. It can be used to determine if the overall organizational objectives of the institution are met, by ensuring that the aggregate RTS ratio of the institution is in the right direction. A research-intensive university may aim for an aggregate RTS Ratio of $R:T:S = 6:4:2$, for example.

It can also be used to determine if the faculty are overloaded with work, by checking if the RTS ratio corresponds to the actual workload self-reported by the faculty. A misalignment between the two can indicate that the faculty are overloaded with work, and can be used to determine if additional faculty need to be hired.

\subsection{Better prioritization of earlier year courses}

Many techniques such as pre-allocation, dynamic splitting, and workload relaxation, needed to be developed to overcome unallocated lectures which were a result of absolute prioritization of earlier year courses. Primitive analysis of allocating all years together showed that many of the unallocated lectures would be allocated if the lectures weren't allocated year-wise. However as shown in tutorial allocation, a year-wise bias was not very effective.

If a better way of prioritizing earlier-year courses can be found, it would be possible to allocate all lectures together, instead of allocating lectures year-wise, which could yield better results. This can be explored in future work.

\subsection{Prioritization of Foundational Courses}

Certain courses are considered to be foundational courses, which are important for the students to learn, and are also important for the students to perform well in, as they are prerequisites for other courses, and play an important part in the students' learning journey. Allocating earlier-year courses to the best faculty broadly serves to prioritize such foundational courses, but it is not very targeted.

An analysis of course prerequisites can be performed, to determine which courses are foundational, and can be given a higher priority in the allocation process. Exploring these interdependencies, such as analyzing the effect of student performance in a foundational course on the student performance in the subsequent courses, can highlight other valuable insights, that can be used to improve the allocation process and learning outcomes for the students.

\subsection{Prioritizing Faculty with Prior Experience in the Course}

As shown in the course workload model, faculty who have previously taught a course have a lower workload than faculty who are teaching the course for the first time. Thus, prioritizing faculty who have previously taught a course can be an additional consideration in avoiding the fragmentation of tutorials and labs. This can also be incorporated into lecture allocation, as it has the potential of lowering overall teaching workload in the institution, and can be explored further.

Additionally, the reduction of workload accompanied by teaching multiple tutorials of the same course is not accounted for in the current model and can be incorporated into the model in future work.

\subsection{Timetabling Constraints}

Timetabling of the courses is carried out separately, independent of the allocation of teaching workload to faculty. However, the two processes are interdependent, and if two courses are timetabled at the same time, they cannot be taught by the same faculty. This can be incorporated into the allocation process, to ensure that the courses are timetabled in a way that allows the courses to be taught by the same faculty.

Other timetabling constraints like the distance between classrooms can also be incorporated into the allocation process, to ensure that faculty have adequate time to reach the classroom after teaching another class. This can be explored in future work.

\subsection{Workforce Gap Analysis}

Techniques like pre-allocation highlighted the shortage of eligible faculty for certain courses. This can be used to perform a workforce gap analysis, which can be used to determine the number of additional faculty that need to be hired to ensure that all courses can be taught by eligible faculty. This, in combination with the need for systematic overloading of faculty, shows the possibility for a more targeted hiring process, which can be beneficial for the institution.
