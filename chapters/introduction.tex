\chapter{Introduction}

This chapter briefly introduces the problem of multi-constraint teaching workload allocation, the motivation for and objectives of the corresponding research. It also describes the structure of the report and the contents of its corresponding chapters.

\section{Background}

Teaching workload allocation and scheduling are some of the most crucial activities in educational institutions of all sizes. Although done once every semester, these activities are the backbone of the university environment and govern how much time faculty members can spend in individual areas of the workload like preparing for lectures, delivering lectures, spending time on research and administrative work. Thus, they have a drastic impact on the academic performance of students, and quality of teaching and research outputs of the faculty.

Historically, this problem has been difficult to automate due to the various conflicting factors that govern the allocations, the massive scale of the problem in large universities like NTU, along with difficulties in accurately quantifying the workload of faculty members. If done poorly, the allocation can lead to faculty dissatisfaction and an inefficient usage of the precious time of students, faculty, and administrative staff.

\section{Motivation}

Although there are multiple in-house techniques and existing solutions to solve this problem, they generally fail to account for the real-world constraints and business decisions that are crucial to the problem. They also struggle with scale, which is important when it comes to large universities like NTU. As a result, weeks of manual effort is wasted on the allocations. The allocations are typically carried forward from previous years, which leads to staff dissatisfaction due to lack of variety in their teaching schedule. This rigidity also makes it tougher to make changes to the curriculum.

Even with the use of existing solutions, workload inequity remains a key problem, with some staff members having much higher workload than others. Along with this, due to a lack of systematic change management approaches, small fluctuations like a faculty leaving the university on a short notice results in makeshift improvisation to grapple with the unavailability, thus inhibiting the learning of students.

\section{Research Outline}

In this chapter, we discussed motivation for improving on the teaching allocation.

\textbf{Chapter 2} offers a review into the recent approaches to Workload Allocation Models (WAMs), and other areas relevant to allocating teaching workload to faculty.

\textbf{Chapter 3} describes the Workload Allocation Model (WAM) proposed to be used in the allocation system, as well as the approach to define the constituent parts of the WAM.

\textbf{Chapter 4} quantifies the teaching workload that needs to be allocated using the WAM, and also describes the approach to the division of said workload.

\textbf{Chapter 5} describes the various factors and constraints that guide the teaching allocation process. It then goes on to define the various inputs required for the allocation, and how all these fit into the allocation algorithm.

\textbf{Chapter 6} gives a high-level view of the system architecture and how parts of previous sections align together to help achieve a good teaching allocation. It then describes the timeline for the research to take place.

% \textbf{Chapter 9} highlights the need for a change-management system in building a resilient system that can account for sudden changes to faculty availability due to unforeseen circumstances.
% \textbf{Chapter 7} describes the factors that need to be considered while choosing the faculty to be allocated for a course, including faculty ratings, course importance, faculty preferences etc.
