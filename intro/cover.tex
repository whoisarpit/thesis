

\begin{titlepage}
  \begin{center}
    \setstretch{2}



    \begin{figure}
      \centering
      \includegraphics[width=0.7\textwidth]{images/Nanyang_Technological_University.png}
    \end{figure}

    \null % Empty line to make vfill work

    \vfill

    {\LARGE\textbf{\thesistitle}}

    \vfill

    {\Large\textbf{\authorname}}\\
    {\Large \textbf{\deptname}\\
    \textbf{\submissionyear}}


  \end{center}
\end{titlepage}
