\chapter*{Abstract}

Allocation of teaching workload has become one of the most important administrative activities of an academic institution with a research-intensive focus. Teaching workload allocations must be inclusive of faculty preferences, teaching performance as well as research and service workloads for ensuring enhanced satisfaction of faculty, students, and management. Despite its significance, the process of teaching workload allocation is often carried out manually, often resulting in sub-optimal and uneven teaching workload allocations.

In this thesis, a systematic approach to quantify the research workload of a faculty has been proposed. It is based on the workload a faculty will endure to manage his/her research team consisting of researchers at all levels. This was found to be more effective than those based on factors such as the quantum of research funds attracted, research papers published, and citation counts. The proposed method accounts for research supervision efforts required in the management of researchers at different levels of seniority. The service workload of a faculty was quantified as a sum of the service duties assigned. A holistic workload model that is inclusive of research, teaching, and service has led to an inclusive approach to establishing the teaching workload of a faculty. The holistic workload model has paved the way for a much-needed inclusive approach to determine a fair and equitable teaching workload for each faculty.

Next, a comprehensive approach was developed to quantify the workload associated with the lecturing of a course. This saliently accounts for several workload components such as preparation, review, delivery, and grading. Moreover, familiarity with the course and the number of students attending the course are considered. Lecture splitting has been introduced to allow more than one faculty to be assigned to a course depicting a significantly high workload to minimize overloading a faculty with the lecturing component. This was very effective for distributing the lecturing workload of large class sizes, and new courses.

A novel technique, based on the Hungarian Algorithm, was proposed for automating the optimal allocation of lectures to faculty members. The allocation process was sensitive to both the faculty preferences and the teaching performance, to maximize faculty satisfaction without compromising on teaching effectiveness. Moreover, the proposed technique incorporates the management preference for allocating the best-performing faculty members to earlier-year courses for enhanced student satisfaction and learning outcomes. Based on the real-life dataset utilized, it was observed that 47\% of the lectures were allocated while preferring faculty members who performed well in the lower-year courses.

Several techniques were proposed and validated for identifying suitable faculty members for lecturing the unallocated courses. Firstly, a pre-emptive allocation technique was introduced to allocate eligible faculty to courses with an acute shortage of faculty supply, prior to the commencement of the main allocation process. This approach has been shown to lower the number of unallocated lectures by 16\%. Three further techniques, namely dynamic splitting, targeted overloading, and selective dynamic swapping have been introduced to resolve the problem of unallocated lectures. To resolve any remaining unallocated courses, an equal misery approach was introduced to overload the teaching workload for the entire faculty cohort. These techniques were found to be highly effective in reducing the number of unallocated lectures and were able to allocate 99.5\% of the lectures for the real-life dataset considered. The proposed techniques provide for a systematic approach for allocating faculty to all lectures despite an acute shortage of faculty or faculty expertise while ensuring that the priorities of all stakeholders are considered appropriately.

To allocate tutorials and labs to faculty members, the proposed techniques based on the Hungarian Algorithm were further extended to incorporate the additional constraints for allocating tutorials and lab sessions to faculty members. Noting that uncontrolled allocation of tutorials and labs may lead to a higher teaching workload due to the involvement in multiple courses, multiple tutorials and lab sessions of a course were grouped to minimize fragmentation. Moreover, a consistency bias was introduced in the Hungarian Algorithm to ensure that a faculty member is assigned tutorials and labs of the course being taught by the faculty. These techniques were shown to reduce the fragmentation from 4.6 faculty members per course to 1.9 without affecting teaching effectiveness. To resolve the problem of unallocated tutorials and labs, the workload limit was relaxed iteratively in a targeted manner until all unallocated tutorials and labs were dealt with while minimizing widespread overloading. It was observed that 100\% of the tutorials and labs were allocated at the expense of overloading 9.6\% of the faculty members, of which only 1.8\% were heavily overloaded.

The above methods and techniques were implemented and validated using real-world data, and the results were found to be highly effective, accurately quantifying the workload in a variety of scenarios, and allocating lectures, tutorials, and labs to faculty members while minimizing fragmentation and uneven faculty overloading. The entire teaching workload allocation process involving 1152 lectures, tutorials, and labs took only 30 seconds on a standard PC. Finally, the proposed techniques ensure an inclusive approach by incorporating the preferences of students, faculty, and management while ensuring an equitable proportion of lectures, tutorials, and labs for each faculty member.

\addcontentsline{toc}{chapter}{\textbf{Abstract}}
