\label{chapter:formal_informal_workload}
% Distributing Teaching Workload

The teaching workload, consisting of formal and informal teaching workload, has some nuances involved in its distribution. This section describes the methodology used for this distribution.

\section{Defining Formal Workload}

\label{section:defining_formal_workload}

Formal Teaching Workload refers to the courses that are taught in a university. A course consists of multiple teaching activities like Lectures, Tutorials and Labs, which need to be quantified. As defined by \parencite{griffith2020framework}, larger classes require more work in non-classroom tasks like grading assignments, office hours etc. A course C can be divided into three sections - the contact hours \(T^c_c\), the administrative work related to the course \(T^a_c\), and the preparatory work \(T^p_c\) .

The preparatory, and contact hours are largely unaffected by class size, while the administrative work heavily depends on the class-size. Course newness and familiarity is something that doesn't affect the contact hours, but has a massive impact on the preparatory work, and has some impact on the administrative work (extra planning and time is required for correction of assignments etc).

For the lecture component, we define the three workloads and the total workload \(T^c\) as

\begin{equation}
\begin{aligned}
	T^c_c &= w*h_w = h  \\
    T^c_p &= h * 3 * n_p \\
    T^c_a &= h * s * n_a \\
    T^c &= T^c_c +T^c_p + T^c_a\\
    T^c &= h(1 + 3 n_p + s*n_a) 
\end{aligned}
\end{equation}
where,
\begin{equation}
\nonumber
	\begin{aligned}
	h_w &= \text{Number of hours of teaching per week}\\
    w &= \text{Number of teaching weeks}\\
    h &= \text{Total number of teaching hours}\\
    s &= \text{Class size factor}\\
    n_a &= \text{Course Novelty Factor - Administration}\\
    n_p &= \text{Course Novelty Factor - Preparation}
   	\end{aligned}
\end{equation}
Tutorials and Labs can only be conducted in fixed-sized groups of about 30-40 each. As a result, although each tutorial/lab session is not affected by the class size directly, the number of sessions increase as the class size increases. Accounting for the rest of the factors of course newness, familiarity etc., we can define the workload of labs and tutorials for every group.

\begin{equation}
\begin{aligned}
	T^c_c &= w*h_w = h  \\
    T^c_p &= h * 3 * n_p \\
    T^c_a &= h * s * n_a \\
    T^c &= T^c_c +T^c_p + T^c_a\\
    T^c &= h(1 + 3 n_p + n_a)
\end{aligned}
\end{equation}

\begin{table}
\centering
\begin{tabular}{|l|l|l|}
	\hline
	Parameter & Constraint & Value \\
    \hline
    \(n_p\) & New to school & 2 \\
     & New to faculty & 1.5 \\
     & Regular & 1 \\
    \hline
    \(n_a\) & New to school & 1.5 \\
     & New to faculty & 1.25 \\
     & Regular & 1 \\
    \hline
\end{tabular}
\caption{Different parameters used in course workload distribution}
\label{course_weightages}
\end{table}

Further exploration and data collection is required to define the exact impact values of the various parameters.

\section{Distributing Formal Workload}
\label{section:distributing_formal_workload}

To distribute the course workload, we use the teaching workload calculated in \autoref{teaching_workload}. However, as seen in \autoref{rts_ratio}, the ratio between lecturers and professors' teaching workload is 2:1. This ratio gets even more skewed as professors get research supervision workload or service duties, which are typically not taken up by lecturers. Thus, in cases of under-supply of faculty expertise, the amount of workload given to lecturers might hit a ceiling, beyond which further work can't possibly be taken up. To resolve this, we allocate in descending order of workload, while a max workload \(C^{max}\) can be given to any given faculty.

To demonstrate, let's take faculty A, B, C, and D, with respective \(T_f\) being 8, 7, 3, and 2. Also, let the total teaching workload be 60 hours and \(C^{max}\) be 20 hours. \(T_{total}= 8+7+3+2 = 20\). So in descending order, Faculty 1 should be given workload \(C_A = 8*60/20=24\), but since 24 > \(C^{max}\), Faculty A will be given \(C_A = 20\). In the same way, B will be given 20. The remainder of workload is equally distributed between C and D as 10 hours each. 

In this way, by using \(C_{max}\), we avoid overloading A and B, by requesting C and D to share the workload of \textbf{10 hours}, instead of the \textbf{6 hours} that they would be originally given.

\section{Distributing Informal Workload}
Informal teaching workload refers to FYPs, Master and PhD students assigned to a faculty. The total informal teaching workload needs to be distributed according to \(T_f\) of the faculty. For this, the total informal teaching workload needs to be distributed among faculties, and then their Masters and PhD workloads subtracted to get the workload available for FYPs. This is because Master and PhD students are typically pre-assigned to the faculty. The exact approach to informal workload distribution, as well as the weekly workload hour impact that each of these three activities, remains to be finalized by looking at the faculty data.