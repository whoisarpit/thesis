
\chapter*{Abstract}

In a research-intensive university, allocation of teaching workload is one of the most important activities of the academic year. It has wide-ranging implications, affecting the learning outcomes of students, the research output of faculty members, faculty attrition and retention, and the overall reputation of the university. Despite its importance, the process of teaching workload allocation is often manual, with a limited scope and a lack of transparency. This leads to an error-prone process that often results in disproportionate workloads for some faculty members.

First and foremost, this thesis proposes a comprehensive approach to quantify the workload involved in teaching a course. This saliently accounts for different stages of teaching a course - preparation, review, delivery, and grading. It is also able to account for various factors that impact the workload, such as faculty familiarity with the course, class size, and the type of activity being performed. The approach was validated using real-world data, and the representation of workload was found to be highly accurate.

A lecture-splitting technique was also developed to reduce the impact of very large class sizes, and new courses, which were found to be a major cause of faculty overloading. The technique was designed to split high-workload lectures into two or more smaller lectures, which were then taught by different faculty members. The technique was found to be highly effective in reducing the workload of faculty members, while also improving the learning experience of students.

It also proposes a systematic approach to quantify the research workload of a faculty, accounting for hierarchical effects involved in research supervision, while avoiding the lagging nature of research output as a metric. In the process of equitably allocating the teaching workload to account for this variation, a workload model was developed to provide a transparent and accurate representation of the faculty workload, taking into account their research, service, and teaching workload. This workload model was then used to determine the teaching workload to allocate to each faculty member, ensuring that the workload was fair and equitable.

A novel technique was proposed using the Hungarian Algorithm to allocate lectures to faculty members. The allocation process incorporated both the faculty preferences and the teaching performance, to maximize faculty satisfaction without compromising on teaching effectiveness and student learning. Moreover, the proposed technique for lecture allocation also considers the management preference of allocating the best-performing faculty members to earlier-year courses. Most importantly, the proposed technique ensures a similar proportion of lectures, tutorials, and labs for each faculty member, further ensuring workload equity.

In the process of allocation, several additional techniques to avoid unallocated lectures and faculty overloading were developed. The first technique involved pre-emptive allocation of courses with an acute shortage of faculty members to teach them. The second technique was Dynamic Splitting, which involved splitting lectures if they remained unallocated due to faculty workload limits. The third technique was Dynamic Swapping, which involved swapping the faculty for certain lectures, in order to allocate other lectures to said faculty.

Finally, an equal misery approach was devised to resolve the problem of unallocated lectures remaining even after the application of the above techniques. This approach involved equitable overloading of the entire faculty cohort, in order to ensure that no faculty member were disproportionately overloaded. With these techniques applied, the allocation was able to allocate 99.5\% of lectures to faculty members, while ensuring that the priorities of all stakeholders were taken into account and faculty workload inequity was avoided.

The allocation of tutorials and labs involved a unique challenge of avoiding fragmentation, i.e. the allocation of a tutorial or lab to multiple faculty members, which results in a poor learning experience for students and a high workload for faculty members. To resolve this, two techniques were devised which in combination were able to avoid fragmentation. The first technique involved batching tutorials and labs, which involved allocating multiple tutorials or labs to a single faculty at a time. The second technique introduced a consistency bias in the Hungarian Algorithm, which preferentially allocated tutorials and labs to faculty members who were already teaching a course. These techniques were able to reduce the number of faculty members teaching a course to less than 50\% of the original. With these techniques, the allocation was able to allocate 100\% of the tutorials and labs to faculty members.

The above methods and techniques were implemented and validated using real-world data, and the results were found to be highly effective, accurately quantifying the workload in a variety of scenarios and alleviating the problem of faculty overloading. The allocation was also found to be highly effective in allocating lectures, tutorials, and labs to faculty members while avoiding problems of fragmentation, and faculty overloading, and reducing the need for manual intervention. Moreover, the proposed techniques were found to be highly scalable, owing to the choice of algorithm, batching of tutorials and labs, and year-wise allocation of lectures. Additionally, the allocation was found to be highly repeatable and deterministic in comparison to other, more heuristic-based approaches.

\addcontentsline{toc}{chapter}{\textbf{Abstract}}
