\chapter{Conclusion and Future Work}

\section{Conclusion}

In this thesis, the problem of automated teaching workload allocation to faculty members was addressed. The proposed solutions were found to be holistic and effective, and were able to allocate all teaching workload to the faculty, while ensuring that the allocation was fair and equitable, and prioritized the student feedback, the faculty preferences, and the management priorities.

To allocate the courses, the problem was split into four sub-problems - determining the workload of each faculty in the domains of research and service to equitably distribute the workload, determining the workload of the teaching activities of the courses, allocating lectures of a course to the faculty, and allocating tutorials and labs of a course to the faculty.

First, this thesis proposed a systematic approach to quantify the research workload of the faculty. This was achieved by identifying the amount of work required by the faculty to manage the research team under their supervision, which was found to be a good representation of the research workload of the faculty. The approach based on the research team size was found to be more accurate than existing approaches based on the number and quality of publications of the faculty, due to difficulties in accurately quantifying the amount of work required to publish a paper, owing to the large variance in the amount of research behind a paper and variance in the quality of journals that the paper is published in. There was a non-linear correlation found between the size of the research team and the research workload of the faculty, owing to a naturally hierarchical structure of staff management, with some staff being directly managed by the faculty, who in turn manage the rest of the research staff. To account for this non-linearity, a non-linear normalization function was used to normalize the research workload of the faculty.

The service workload was measured based on the individual service activities which were manually rated by the management for their workload, the service workload of a faculty being the sum of all the service activities that the faculty is involved in. The research and service workload was then modelled into an RTS ratio, which provided a holistic representation of the workload constitution of a faculty. This provided much-needed transparency into the workload of the faculty for management and faculty alike and was found to be a good representation of the workload of the faculty in various areas. The RTS ratio was then used to equitably determine the relative teaching workload of the faculty, accounting for the research and service workload of the faculty.

Next, a systematic approach was proposed to quantify the workload of the teaching activities of a course. To quantify the workload of the teaching activities of a course, the activity was divided into four distinct constituents, namely - preparation, review, delivery, and grading workload. Moreover, the faculty's familiarity with the course and the class size was considered to quantify the workload. The workload of each of the teaching constituents was then modelled individually for lectures, tutorials and lab sessions, in the form of workload units that accounted for the time and effort required to perform the teaching activity. Testing the model with real-world data, the model was found to be a good representation of the workload of the teaching activities of a course and was found to be effective in quantifying the workload of the teaching activities of a course in a variety of scenarios. The workload of teaching activities, in conjunction with the RTS ratio of the faculty, was then used to determine the teaching workload limits for the faculty members, which was used in the allocation process to ensure that the faculty members were not overloaded with work.

Using this model, the workload of lectures was found to be anywhere from 150 to 700 units, depending on the class size and the familiarity of the faculty with the course. This high variance posed the risk of overloading the faculty with work, and thus, a lecture splitting algorithm was developed, that splits the workload of a lecture into 2-3 smaller lectures, which can then be allocated to multiple faculty members. This was very useful for distributing the lecturing workload of large class sizes, and new courses and the technique was found to be highly effective in reducing the workload of faculty members, while also improving the learning experience of students.

A novel technique, based on the Hungarian Algorithm, was proposed for automating the optimal allocation of lectures to faculty members. The allocation process was sensitive for both the faculty performance for the lectures in previous years and the faculty preference for the courses in order to maximize faculty satisfaction, without compromising on the teaching effectiveness of the course. The allocation process was structured such that the teaching workload limits of the faculty were not exceeded, and the faculty members were not overloaded with work. It also ensured that the management priority of ensuring that earlier-year courses got preferential allocation of the best faculty was met, to ensure a good learning experience of the students in the foundational courses. This allocation algorithm was able to allocate 46\% of the lectures to the faculty members. It was also observed that the preferences of the faculty were honoured, with more than half of the faculty teaching lectures of their top 5 preferences.

Several techniques were proposed and validated to allocate faculty to the unallocated lectures which remained after the allocation process. First, a pre-emptive allocation technique was introduced, which identified an acute shortage of faculty supply and expertise for certain lectures, and pre-emptively allocated them to the faculty before the allocation process. This was found to reduce the number of unallocated lectures by 15\%. Next, a dynamic splitting technique was introduced, which identified the lack of faculty bandwidth for teaching unallocated courses, and solved this by allocating two or more faculty to teach the lecture, resulting in a lower impact on their workload. This was found to be very effective, independently reducing the number of unallocated lecture hours by 24.4\%.  A targeted overloading technique was introduced, which overloaded the faculty members needed for allocating unallocated lectures, with the foresight that tutorial and lab allocation would underload the faculty members to compensate for the inequities created. This was found to be very effective, independently reducing the number of unallocated lecture hours by 35.8\%. A dynamic swapping technique was introduced, which attempted to free up faculty members for teaching unallocated lectures, by swapping their lectures with other faculty while ensuring that the teaching performance was not compromised. This was able to independently reduce the number of unallocated lecture hours by 15.6\%.

A combination of all these techniques was able to allocate 97.5\% of the lectures to the faculty members. Finally, an equal misery approach was introduced, which was able to allocate 99.5\% of the lectures to the faculty members by overloading all the faculty members equally. The allocation process, in combination with these techniques, was able to allocate the lectures to faculty members in an automated manner, accounting for various challenges that different datasets might pose, thus reducing the need for manual intervention.

To allocate tutorials and labs to faculty members, the proposed techniques based on the Hungarian Algorithm were adapted to account for additional challenges unique to tutorials and lab sessions. Uncontrolled allocation of tutorials and labs to faculty members can result in fragmentation i.e., a large number of faculty members teaching the tutorials and labs of the same course. This can be detrimental to the learning experience of the students due to the lack of consistency in teaching across the student cohort, while also increasing the workload overhead for faculty members. To address this, a consistency bias was introduced, which favoured the faculty members who were already teaching lectures, tutorials or labs of the same course, to ensure consistency in teaching. This was found to be very effective, reducing the number of faculty teaching the tutorials and labs of the same course from 4.6 faculty teaching tutorials and labs of a course to 3.8 faculty. Next, a batching technique was introduced, which allocated tutorials and labs in batches of 6 sessions at a time, reducing fragmentation. This was able to further reduce the number of faculty teaching the tutorials and labs of the same course to 1.9 faculty.

To account for unallocated tutorials and labs, the pre-emptive allocation of tutorials was introduced, which was able to reduce the number of unallocated tutorials by 3\%. A selective workload relaxation technique was introduced, which identified the workload limits causing the unallocated tutorials and labs, and relaxed the workload limits of the faculty members to allow them to teach the unallocated tutorials and labs. This was found to be very effective, reducing the number of unallocated tutorials and labs from 27\% to 0\% without adversely affecting workload equity and fairness. The allocation process, in combination with these techniques, was able to allocate 100\% of the tutorials and labs to the faculty members in an automated manner, with only 1.8\% of the faculty members being heavily overloaded, while ensuring consistency of student experience.

Implementing and validating the above methods with real-world data revealed them to be highly flexible in allocating lectures, tutorials and labs to the faculty, with the ability to accommodate the unique challenges that different datasets might pose. The allocation process was able to satisfy the requirements of students, faculty and management while avoiding problems like widespread faculty overloading and fragmentation of teaching. Unlike existing heuristic-based approaches, the proposed techniques were also found to be highly deterministic and scalable, with the ability to allocate the activities of an entire semester involving 1152 lectures, tutorials and labs, within 30 seconds on a standard PC. Finally, the allocation process was found to be equitable, accounting for the workload of faculty in non-teaching areas and ensuring that these areas are accounted for in the teaching allocation process.


\section{Future Work}

The allocation of teaching workload to faculty is a complex problem, and while the allocation achieved in this thesis is fair and equitable, there are still certain areas of improvement that can be made. Some of these areas are discussed below.

\subsection{Relaxing prioritization of earlier year courses to minimize unallocated lectures}

The absolute prioritization of earlier-year courses, while effective in ensuring that the foundational courses are taught by the best faculty, can result in a large number of unallocated lectures, as the best faculty who are teaching the earlier-year courses may be essential to teach the later-year courses. While this thesis proposed several techniques to handle this which proved to be effective in resolving these unallocated lectures, relaxing the prioritization of earlier-year courses can be explored in future work.

The prioritization of earlier-year courses was also performed to prioritize foundational courses which form the basis of the later-year courses, and are important for the students to learn and perform well in. An analysis of the course prerequisites can be performed, to determine which courses are foundational, and can be given a higher priority in the allocation process instead of a blanket prioritization of earlier-year courses.

Exploring the course interdependencies, such as analyzing the effect of student performance in a foundational course on the student performance in the subsequent courses, can highlight other valuable insights, that can be used to improve the allocation process and learning outcomes for the students.

\subsection{Accounting for the impact of Teaching Assistants}

Teaching Assistants (TAs) are typically graduate students who assist the faculty in teaching the course. Their role is to assist the faculty in teaching activities, such as grading, and in some cases, they may also be involved in the delivery of the tutorials and labs. Their presence might help in reducing the workload of the faculty, and can also be used to improve the learning outcomes of the students by providing them with additional support. The impact of TAs on the teaching workload of the faculty is not accounted and thus, adapting the proposed models for TA supply and involvement would be a worthy pursuit.

\subsection{Allocation of Final Year Projects}

Final Year Projects (FYPs) are a capstone project that is undertaken by the students in their final year of study. The FYPs are supervised by a faculty member, and the faculty member is responsible for the grading of the FYP. The workload of the FYP is considered as teaching workload, and thus, the allocation of FYPs to the faculty is an important consideration in the allocation process. The RTS ratio can be used to determine the number of FYPs that can be allocated to the faculty, and the allocation process can be adapted to allocate the FYPs to the faculty. An exploration in this area will make the allocation process more comprehensive and holistic.

\subsection{Timetabling Constraints}

Timetabling of the courses is carried out separately, independent of the allocation of teaching workload to faculty. However, the two processes are interdependent, and if two courses are timetabled at the same time, they cannot be taught by the same faculty. This can be incorporated into the allocation process, to ensure that the courses are timetabled in a way that allows the courses to be taught by the same faculty.

Other timetabling constraints like the distance between classrooms can also be incorporated into the allocation process, to ensure that faculty have adequate time to reach the classroom after teaching another class. This can be explored in future work.

\subsection{Workforce Gap Analysis}

Techniques like pre-allocation highlighted the shortage of eligible faculty for certain courses. This can be used to perform a workforce gap analysis, which can be used to determine the number of additional faculty that need to be hired to ensure that all courses can be taught by eligible faculty. A closer examination of the degree of deviation from faculty preference should pave the way for identifying a shortage of expertise in certain areas. This, in combination with the need for systematic overloading of faculty, shows the possibility for a more targeted hiring process, which can be beneficial for the institution.

\subsection{Inter-semester workload balancing}

It was seen that certain faculty members needed to be overloaded in a semester to ensure the allocation of all teaching workload to the faculty. There is scope for inter-semester workload balancing, where the workload of the faculty is balanced across semesters, to ensure that in the event of an overloading of faculty in one semester, the workload of the faculty is reduced in the subsequent semester.

\subsection{Realization of Institutional Goals}

The RTS model provided a transparent and insightful metric for the workload of the faculty. This can be used to determine if the institutional goals are being met, by ensuring that the aggregate RTS ratio of the institution is in the right direction. A research-intensive university may aim for an aggregate RTS Ratio of $R:T:S = 6:4:2$, for example. The RTS ratio can also be used to determine if these goals are being met at the department level, and faculty level, and can be used to identify areas of improvement.
