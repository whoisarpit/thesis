
\begin{abstract}

  Faculty workload allocation and scheduling are key factors contributing to the quality of research, teaching and service in large educational institutions. This necessitates the development of suitable workload models for research, teaching and service to facilitate the equitable distribution of workload among research-active faculty. In this research work, a suitable workload model for evaluating the research workload of a faculty has been proposed based on the attributes of the research group of a faculty. Teaching workload model relies on formal and informal teaching components as well as weighted association among lectures, tutorials, and labs. A novel technique for accommodating large class sizes based on available teaching expertise has also been proposed. The main aim of this research is to realize an automated teaching allocation system that can incorporate faculty preferences and teaching priorities to achieve an inclusive, transparent, and objective distribution of teaching workloads while improving the overall faculty satisfaction levels.

\end{abstract}
