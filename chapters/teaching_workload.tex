\chapter{Computing Teaching Workload}

\section{Introduction}

To allocate teaching workload in a fair and transparent manner, the first step is to accurately quantify the teaching workload to ensure that important parts of the teaching workload are not overlooked. Previous attempts at computing the teaching workload have been made, but they have been largely retrospective in nature, looking at the activities performed by faculty in the past to compute the teaching workload. However, for allocating a teaching workload, a prospective approach is required, computing the teaching workload as a function of the activities that need to be performed and the factors that affect them.

Teaching activities have previously been classified into the buckets of classroom contact hours (delivery of lectures etc), course administration workload (grading examinations etc) and preparation workload (preparing and revising course material) \cite{griffith2020framework}. However, a need to further divide the teaching workload into more granular activities was felt. This was done to better account for the distribution of workload between the different faculty involved in teaching a course - course lecturers, tutors and lab instructors. An analysis of various courses offered at the \deptname revealed that the teaching workload could be divided into granular activities. These include preparing the course material, delivering the course material, grading the examinations and assignments, and consultation. In the following sections, we will discuss the various activities involved in teaching a course and how they are distributed between the different faculty involved in teaching a course.

\section{Computing Lecture Workload}
\label{sec:workload_of_course_lecturers}

Course Lecturers take the primary responsibility of managing a course. On top of delivering the lectures, they are also responsible for preparing the course material, defining a rubric for the course, designing the examinations and assignments, and grading them. For certain courses, the workload may be too high for a single faculty to handle. In such cases, additional course lecturers may be assigned and the responsibility of delivering the course is split into equal durations of the term. For example, if a 13-week course is taught by a 2 course lecturers, they will be responsible for delivering the course for 7 and 6 weeks respectively. They also share equal responsibility for preparing the course material as well as designing the examinations and assignments for their respective parts, and split the workload of grading the examinations equally. The activities involved in teaching a course can be distributed into three phases - before the course starts, during the course and after the course ends. The activities involved in each of these phases are listed below.

\begin{enumerate}
  \item Before the course starts
        \begin{enumerate}
          \item Preparation of lecture material
          \item Preparation of tutorial material
          \item Preparation of lab material
        \end{enumerate}

  \item During the course
        \begin{enumerate}
          \item Lecture Delivery
          \item Consultation
        \end{enumerate}

  \item After the course ends
        \begin{enumerate}
          \item Grading Examinations
          \item Grades Entry and Moderation
        \end{enumerate}
\end{enumerate}

\subsection{Preparation of Course Material}
\label{sec:preparation_of_course_material}

Preparation of course material involves a significant amount of research and preparation, designing up-to-date contents for the course, time spent designing lecture slides, time spent designing an evaluation strategy for the course and accordingly, defining the rubric for the course and designing the examinations and assignments. Additionally, even though tutorials and lab sessions are delivered by tutors and lab instructors respectively, the course lecturer has a significant role in preparing the tutorial and lab material. This includes designing the tutorial material and lab experiments, as well as the solutions to the tutorial and lab questions. The following factors may have an impact on the preparation workload:

\begin{enumerate}
  \item \textbf{Class Size} - No Impact

        Since the preparation of course material is a one-time activity conducted at the start of the course, the class size has minimal impact on the preparation workload. It could be argued that the course material needs to be prepared more thoroughly for a larger class size, but the impact of such factors is assumed to be minimal

  \item \textbf{Course Newness} - Significant Impact

        The familiarity of the faculty with the course material can have an impact on the preparation workload. If the faculty is unfamiliar with the course material, they will need to spend more time researching and preparing the course material. However, another faculty might have previously taught the same course, and thus the course material is readily available. Thus, the preparation workload can be divided into three distinct scenarios:

        \begin{enumerate}
          \item Course Coordinator has previously taught the course\\
                If the course coordinator has previously taught the course, they only need to ensure that the course material is up to date. Thus, the time required to prepare the course material is negligible.
          \item Course Coordinator has not previously taught the course, but another faculty has\\
                If another faculty in the department has previously taught the course, the course material prepared by them is readily available. Most time is spent on reviewing the course material and tailoring it to the course coordinator's teaching style. Thus, the time required to prepare the course material is moderate. Upon looking into previous cases of course preparation, it was found that the preparation workload is reduced by 60\% if another faculty has previously taught the course.
          \item Course Coordinator has not previously taught the course, and no other faculty has either\\
                If no other faculty in the department has previously taught the course, the course material needs to be prepared from scratch. This requires a significant amount of time and effort. Especially for advanced and novel courses, the research required to prepare the course material can be significant. Thus, the time required to prepare the course material is substantial.
        \end{enumerate}

  \item \textbf{Course Complexity} - No Impact

        It can be argued that the preparation workload can vary depending on the course complexity, since the amount of research for complex courses is significantly higher. However, the faculty is typically assigned to teach a course according to their expertise, and thus it can be assumed that the faculty is familiar with the subject matter. This factor is thus not considered in the teaching workload.

\end{enumerate}

\subsubsection{Defining the course material preparation workload (\(T_p\))}

Given the above association of preparation workload, we need to define the workload as a function of the number of hours of lectures, tutorials, and labs, and the familiarity of the faculty with the course material.

We found that preparation of lecture materials for a completely new course takes approximately 10 hours per hour of lecture. Similarly, preparation of tutorial materials for a completely new course takes approximately 5 hours per hour of tutorial. Finally, preparation of lab materials for a completely new course takes approximately 2.5 hours per hour of lab. So for a new course, we can say:

\begin{equation}
  \label{eq:preparation-workload}
  \begin{aligned}
    T_p^{lec} & = 10  & \times H^{lec} \\
    T_p^{tut} & = 5   & \times H^{tut} \\
    T_p^{lab} & = 2.5 & \times H^{lab}
  \end{aligned}
\end{equation}

where \(T_p^{lec}\), \(T_p^{tut}\), and \(T_p^{lab}\) are the preparation workload for lecture, tutorial, and lab respectively, and \(H^{lec}\), \(H^{tut}\), and \(H^{lab}\) are the number of hours of lectures, tutorials, and labs respectively.

In \autoref{sec:preparation_of_course_material}, we found that the preparation workload is affected by the familiarity of the course coordinator with the course material.  that if the another faculty has previously taught the course, the preparation workload is reduced by 60\%. We also found that if the same course coordinator has previously taught the course, the preparation workload is negligible. To account for these, we introduce \textbf{Newness to School} (\(N_1\)) and \textbf{Newness to Faculty} (\(N_2\)) factors. These factors are defined in \autoref{tab:n1-workload-factor} and \autoref{tab:n2-workload-factor} respectively. The need to introduce two separate factors is felt because in certain scenarios, the workload of a course independent of the faculty needs to be known. For example, if the total workload for the school needs to be calculated, the workload of a course needs to be calculated independent of the faculty. In such a case, the \(N_1\) is used. However, if the workload of a faculty needs to be calculated, both \(N_1\) and \(N_2\) need to be considered.

\begin{table}[ht]
  \centering
  \begin{tabular}{|l|c|}
    \hline
    \textbf{Case}                 & \textbf{\(N_1\)} \\ \hline
    Course is new to school       & 1                \\ \hline
    Course has been taught before & 0.4              \\ \hline
  \end{tabular}
  \caption{\(N_1\) - Newness to School factor}
  \label{tab:n1-workload-factor}
\end{table}

\begin{table}[ht]
  \centering
  \begin{tabular}{|l|c|}
    \hline
    \textbf{Case}                        & \textbf{\(N_2\)} \\ \hline
    Course is new to Faculty             & 1                \\ \hline
    Faculty has taught the course before & 0                \\ \hline
  \end{tabular}
  \caption{\(N_2\) - Newness to Faculty factor}
  \label{tab:n2-workload-factor}
\end{table}

For simplicity, we define the combined newness factor \(N = N_1 \times N_2\). Thus for the three cases of familiarity, the newness factors can be calculated as follows:

\begin{enumerate}
  \item Course is new to school and faculty has not taught the course before       \\
        \(N_1 = 1\) ; \(N_2 = 1\) ; \(N = 1\)
  \item Another faculty has taught the course before \\
        \(N_1 = 0.4\) ; \(N_2 = 1\) ; \(N = 0.4\)
  \item Course Coordinator has taught the course before     \\
        \(N_1 = 0.4\) ; \(N_2 = 0\) ; \(N = 0\)
\end{enumerate}

So, the preparation workload for a course is given by:

\begin{equation}
  \begin{aligned}
    T_p^{lec} & = 10  & \times N \times H^{lec} \\
    T_p^{tut} & = 5   & \times N \times H^{tut} \\
    T_p^{lab} & = 2.5 & \times N \times H^{lab}
  \end{aligned}
\end{equation}

In total, we can say

\begin{equation}
  \label{eqn:preparation-workload-total}
  \begin{aligned}
    T_p & = T_p^{lec} + T_p^{tut} + T_p^{lab}            \\
        & = (10\ H^{lec} + 5\ H^{tut} + 2.5\ H^{lab})\ N
  \end{aligned}
\end{equation}

\subsection{Lecture Delivery}

Lecture delivery is the time spent by the faculty in delivering the lectures. It may also include some time spent reviewing the lecture material, however since the lecture material is prepared by the faculty themselves, this time is assumed to be minimal. Given this, we can consider how the lecture delivery workload is affected by various factors:

\begin{enumerate}
  \item \textbf{Class Size} - Minimal Impact \\
        Since the lecture delivery workload is a function of the number of hours of lectures, the class size has minimal impact on the lecture delivery workload. It could be argued that teaching a larger class size requires more effort since the faculty needs to ensure comprehension and engagement of the students, but the impact of such factors is assumed to be minimal.

  \item \textbf{Course Newness} - No Impact \\
        Since the course material is already prepared, it can be assumed that by this point the faculty is familiar with the course material. Thus, the course newness has no impact on the lecture delivery workload.

  \item \textbf{Course Complexity} - No Impact \\
        Similar to the course newness, it can be assumed that the faculty is familiar with the course material by this point. Thus, the course complexity has no impact on the lecture delivery workload.

\end{enumerate}

\subsubsection{Defining the lecture delivery workload (\(T_d^{lec}\))}

Given the above association of lecture delivery workload, we need to define the workload as a function of the number of hours of lectures. Since this workload is independent of the class size, course newness, and course complexity, we can simply define the lecture delivery workload as the number of hours of lectures. So, the lecture delivery workload for a course is given by:

\begin{equation}
  \label{eqn:lecture-delivery-workload}
  T_d^{lec} = H^{lec}
\end{equation}


\subsection{Exam Grading}

The exam grading workload is the time required to grade the exams - this includes grading the mid-semester exams, final exams, and any additional assignments that may be given. Although some assistance may be provided by teaching assistants, the faculty is primarily responsible for grading the exams. Given this, we can consider how the exam grading workload is affected by various factors:

\begin{enumerate}
  \item \textbf{Class Size} - Significant Impact \\
        Since the exam grading has to be done individually for each student enrolled in the course, the class size has a significant impact on the exam grading workload. Although some workload is divided amongst teaching assistants, the faculty is primarily responsible for grading the exams. Thus, the exam grading workload is directly proportional to the number of students enrolled in the course.

  \item \textbf{Course Newness} - No Impact \\
        Since the examination material and the grading strategy is already prepared, grading the examinations and assignments is clerical in nature and thus, the course newness has no impact on the exam grading workload.

  \item \textbf{Course Complexity} - No Impact \\
        Similar to the course newness, it can be assumed that all effects of the course complexity are accounted for in the course material preparation. Thus, the course complexity has no impact on the exam grading workload.

\end{enumerate}

Given the above factors, we can define the exam grading workload as a function of the number of students enrolled in the course. However, since the exam grading workload is clerical in nature, we need to account for the fact that an hour spent on grading is less taxing than an hour spent on teaching. This is further discussed in the following section.

\subsubsection{Accounting for Cognitive Load}
\label{sec:accounting_for_cognitive_load}

Although the amount of time spent in various teaching activities might be comparable, the cognitive load associated with these activities can vary significantly. The cognitive effort associated with preparing the course material is significant, since the faculty has to research and prepare the course material to a degree that the tutorials and labs can be delivered by independent tutors and lab instructors. Delivering a lecture also requires a significant amount of effort, since the faculty needs to ensure comprehension and engagement of the students, and engage in realtime feedback to adjust the pace of the lecture. On the other hand, since the examination material and the grading strategy is already prepared, grading the examinations and assignments requires minimal effort. Thus, the cognitive load associated with each of these activities can be ranked as follows:

\begin{enumerate}
  \item \textbf{Course Material Preparation}
  \item \textbf{Lecture Delivery}
  \item \textbf{Examination and Assignment Grading}
\end{enumerate}

As previously described, it can be argued that the cognitive load associated with course material preparation can vary depending on the course complexity since complex courses require higher research, but this is accounted for since the faculty is typically assigned to teach a course according to their expertise. Similarly, teaching a larger class size can also increase the cognitive load associated with lecture delivery, but the impact is assumed to be minimal.

However, because exam grading is a largely clerical task, we need to account for the fact that an hour spent on grading is less taxing than an hour spent on teaching. We found that the cognitive load associated with exam grading is \( 1/4^{th} \) the cognitive load associated with other teaching activities, and thus we introduce a normalization factor \(N_g = 0.25\) to account for this.

\subsubsection{Defining the exam grading workload (\(T_g\))}

We found that on an aggregate, a faculty spends 2 hours per student per semester on grading. So, the exam grading workload for a course is given by \(T_g = 2 \times S\), where \(S\) is the number of students enrolled in the course. So, the exam grading workload for a course is given by \(T_g = 2 \times S\), where \(S\) is the number of students enrolled in the course.
As discussed in \autoref{sec:accounting_for_cognitive_load}, we found that the cognitive load associated with exam grading is \( 1/4^{th} \) the cognitive load associated with other teaching activities. So, we introduce a normalization factor \(N_g = 0.25\) to account for this. So, the exam grading workload for a course is given by:


\begin{equation}
  \label{eqn:exam-grading-workload}
  \begin{aligned}
    T_g & = N_g \times 2 \times S  \\
        & = 0.25 \times 2 \times S \\
    T_g & = 0.5\ S
  \end{aligned}
\end{equation}


\section{Computing Tutorial and Lab Workload}

Tutorials and lab sessions for a course are taught in groups of 30-40 students. Tutors and Lab Instructors are primarily responsible for the delivery of these tutorials and labs respectively. They are also responsible for the grading of tutorial assignments and lab reports, although this activity is largely carried out as part of the delivery of materials, and thus doesn't need to be separately accounted for. Ideally, the course coordinator are given the responsibility of delivering these tutorials and labs. However, it is assumed that the materials for tutorial and lab sessions are prepared to a degree that minimal effort is required to deliver them. With that perspective, the activities involved the teaching workload of tutors and lab instructors can be divided into three phases -

\begin{enumerate}
  \item Before the session: \textbf{Tutorial/Lab Material Review}
  \item During the session: \textbf{Tutorial/Lab Delivery}
  \item After the session: \textbf{Consultation}
\end{enumerate}

\subsection{Material Review}
\label{sec:material_review}

The review workload is the time required to review the materials for tutorial and lab. This includes reviewing the tutorial and lab questions, and the solutions to the tutorial and lab experiments. We can consider how the review workload is affected by various factors:

\begin{enumerate}
  \item \textbf{Class Size} - No Impact \\
        Since the tutorials and labs are delivered in groups of 30-40 students, the class size remains largely constant between different courses. Thus, the class size has no impact on the review workload.

  \item \textbf{Course Newness} - Significant Impact \\
        The familiarity of the faculty with the course material can have an impact on the review workload. If the faculty is unfamiliar with the course material, they will need to spend more time reviewing the course material. We can use the same three scenarios as defined in \autoref{sec:preparation_of_course_material} for the material review workload as well.

  \item \textbf{Course Complexity} - No Impact \\
        It can be argued that the review workload can vary depending on the course complexity, since the amount of familiarization required for complex courses is higher. However, similar to lectures, the faculty is typically assigned to teach a course according to their expertise, and thus this factor is assumed to have no impact on the review workload.

\end{enumerate}


\subsubsection{Defining the material review workload (\(T_r\))}

For every hour of tutorial/lab, we found that a faculty spends 1 hour on reviewing the materials. As discussed in \autoref{sec:material_review}, the review workload is affected by the familiarity of the faculty with the course material. We can use the same newness factors and the same three scenarios as defined in \autoref{sec:preparation_of_course_material} for the material review workload as well. So, the material review workload for a course is given by:

\begin{equation}
  \label{eqn:review-workload}
  \begin{aligned}
    T_r^{tut} & = 1   & \times N \times H^{tut} \\
    T_r^{lab} & = 0.5 & \times N \times H^{lab}
  \end{aligned}
\end{equation}

\subsection{Tutorial/Lab Delivery}

The delivery workload is the time required to deliver the course. This includes delivering the lectures, tutorials, and labs. We can consider how the delivery workload is affected by various factors:

\begin{enumerate}
  \item \textbf{Class Size} - No Impact \\
        As discussed in \autoref{sec:material_review}, the tutorials and labs are delivered in groups of 30-40 students. Thus, the delivery workload is independent of the class size.

  \item \textbf{Course Newness} - No Impact \\
        Since the course material is already prepared, it can be assumed that by this point the faculty is familiar with the course material. Thus, the course newness has no impact on the delivery workload.

  \item \textbf{Course Complexity} - No Impact \\
        Similar to the course newness, it can be assumed that the faculty is familiar with the course material by this point. Thus, the course complexity has no impact on the delivery workload.

\end{enumerate}

\subsubsection{Defining the delivery workload (\(T_d\))}

Since the delivery of teaching activities is completely independent of the class size and course familiarity, the delivery workload is the same as the number of hours of lectures, tutorials, and labs. So, the delivery workload for a course is given by:

\begin{equation}
  \label{eqn:delivery-workload}
  \begin{aligned}
    T_d^{lec} & = H^{lec} \\
    T_d^{tut} & = H^{tut} \\
    T_d^{lab} & = H^{lab}
  \end{aligned}
\end{equation}


\begin{table}[ht]
  \centering
  \begin{tabular}{|l|c|c|c|}
    \hline
    \textbf{Activity}             & \textbf{Class Size} & \textbf{Newness} & \textbf{Effort} \\ \hline
    \multicolumn{4}{|l|}{\color{gray}Course Coordinator}                                     \\\hline
    Lecture Material Preparation  & -                   & \checkmark       & High            \\ \hline
    Tutorial Material Preparation & -                   & \checkmark       & High            \\ \hline
    Lab Material Preparation      & -                   & \checkmark       & High            \\ \hline
    Lecture Delivery              & -                   & -                & Medium          \\ \hline
    Consultation                  & \checkmark          & -                & Insignificant   \\ \hline
    Exam Grading                  & \checkmark          & -                & Low             \\ \hline
    Grades Entry and Moderation   & -                   & -                & Insignificant   \\ \hline

    \multicolumn{4}{|l|}{\color{gray}Tutor/Lab Instructor}                                   \\\hline
    Material Review               & -                   & \checkmark       & Medium          \\ \hline
    Delivery                      & -                   & -                & Medium          \\ \hline
    Consultation                  & -                   & -                & Insignificant   \\ \hline
  \end{tabular}
  \caption{Teaching Workload Factors}
  \label{tab:teaching-workload-factors}
\end{table}

\autoref{tab:teaching-workload-factors} describes the various factors that affect the teaching workload and how they impact the teaching workload.

\section{Total Teaching Workload}

The total workload for a course is the sum of the preparation workload, delivery workload, and exam grading workload. So, the total workload for the course coordinator is given by:

\begin{equation}
  \label{eqn:total-workload}
  \begin{aligned}
    T & = T_p + T_d + T_g                                                                                                                   \\
      & = T_p^{lec} + T_p^{tut} + T_p^{lab} + T_d^{lec} + T_d^{tut} + T_d^{lab} + T_g                                                       \\
      & = 10 \times N \times H^{lec} + 5 \times N \times H^{tut} + 2.5 \times N \times H^{lab} + H^{lec} + H^{tut} + H^{lab} + 0.5 \times S
  \end{aligned}
\end{equation}
where,
\begin{equation}
  \nonumber
  \begin{aligned}
    N       & = newness\ factor                     \\
    H^{lec} & = hours\ of\ lectures                 \\
    H^{tut} & = hours\ of\ tutorials                \\
    H^{lab} & = hours\ of\ labs                     \\
    S       & = students\ enrolled\ in\ the\ course
  \end{aligned}
\end{equation}

For tutors and lab assistants, the workload is the sum of the review workload and delivery workload. So, the total workload for a tutor or lab assistant is given by:

\begin{equation}
  \label{eqn:total-workload-tut-lab}
  \begin{aligned}
    T^{tut} & = T_r^{tut} + T_d^{tut}                 \\
            & = 1 \times N \times H^{tut} + H^{tut}   \\
    T^{lab} & = T_r^{lab} + T_d^{lab}                 \\
            & = 0.5 \times N \times H^{lab} + H^{lab}
  \end{aligned}
\end{equation}
where,
\begin{equation}
  \nonumber
  \begin{aligned}
    N       & = newness\ factor      \\
    H^{tut} & = hours\ of\ tutorials \\
    H^{lab} & = hours\ of\ labs
  \end{aligned}
\end{equation}

Other activities like consultation hours, office hours, and grades entry and moderation are not included in the workload calculation because they are not significant enough to affect the workload distribution.

\subsection{Demonstration of the model}
\label{sec:workload_demo}

For the purposes of demonstration, we will consider three examples:

\begin{table}[ht]
  \centering
  \begin{tabular}{|l|c|c|c|c|c|}
    \hline
    \textbf{Course}                        & \(H^{lec}\) & \(H^{tut}\) & \(H^{lab}\) & \(S\) & \(N\) \\\hline
    Computer Organisation \& Architecture  & 26          & 13          & 26          & 720   & 0.4   \\\hline
    Artificial Intelligence in Game Design & 39          & 0           & 0           & 26    & 0.4   \\\hline
    Data Structures                        & 26          & 13          & 26          & 25    & 1     \\\hline
  \end{tabular}
\end{table}

For the first example, the total workload for the course coordinator is given by:

\begin{equation}
  \nonumber
  \begin{aligned}
    T_1 & = T_p + T_d + T_g                                                                                                                         \\
        & = (10 \times N \times H^{lec} + 5 \times N \times H^{tut} + 2.5 \times N \times H^{lab}) + (H^{lec} + H^{tut} + H^{lab}) + (0.5 \times S) \\
        & = (10 \times 0.4 \times 26 + 5 \times 0.4 \times 13 + 2.5 \times 0.4 \times 26) + (26 + 13 + 26) + (0.5 \times 720)                       \\
        & = 156 + 65 + 360                                                                                                                          \\
        & = 581\ units
  \end{aligned}
\end{equation}

For the second example, the total workload for the course coordinator is given by:

\begin{equation}
  \nonumber
  \begin{aligned}
    T_2 & = 10 \times 0.4 \times 39 + 5 \times 0.4 \times 0 + 2.5 \times 0.4 \times 0 + 39 + 0 + 0 + 0.5 \times 26 \\
        & = 156 + 39 + 13                                                                                          \\
        & = 208\ units
  \end{aligned}
\end{equation}

For the third example, the total workload for the course coordinator is given by:

\begin{equation}
  \nonumber
  \begin{aligned}
    T_3 & = 10 \times 1 \times 26 + 5 \times 1 \times 13 + 2.5 \times 1 \times 26 + 26 + 13 + 26 + 0.5 \times 26 \\
        & = 390 + 65 + 13                                                                                        \\
        & = 468\ units
  \end{aligned}
\end{equation}

As we can see with the third example, a new course can have a substantial amount of workload associated with it, even if the number of students enrolled in it is small. This is because the preparation workload, at 390 units, is much higher for the new course.

\section{Lecture Splitting}

A course with a disproportionately high workload can lead to issues in maintaining an equitable workload distribution in multiple ways. This can be seen in the following scenarios:

\begin{enumerate}

  \item \textbf{Research Active Faculty}

        A faculty with a high research workload might not have the bandwidth to teach a course with a high teaching workload. As a result, even if a faculty has expertise in a particular course's subject matter, they might not be able to teach the course due to a lack of bandwidth. They will instead be allocated a course with a lower teaching workload, which might not be in their area of expertise. This can lead to a sub-optimal allocation of courses.

  \item \textbf{Foundational Courses}

        Foundational courses like Data Structures and Algorithms, Computer Organisation and Architecture, etc. are required by multiple programs. As a result, these courses have a high number of students enrolled in them. However, the inability to allocate an experienced faculty to teach these courses can lead to a sub-optimal learning outcomes for the students which can have a cascading effect on the students' performance in subsequent courses.

  \item \textbf{Inefficiencies in Allocating Multiple Courses}

        If a faculty is allocated a course with a high workload, it becomes difficult to allocate a second course to the faculty. This is because the workload of the faculty is already high, and allocating a second course can lead to unfairly high workload for the faculty. On the other hand, if they're not allocated a second course, the faculty is underutilized, as well as creating supply-demand mismatches in the allocation of courses.

\end{enumerate}

It is observed that the workload for course lecturers is disproportionately high for certain courses. Thus, a way to pre-empt such workload inequities is to split the lecture workload of such courses between two or more faculty, which ensures that all faculty have a comparable workload. This process of splitting the workload of a course between two or more faculty is called \textit{Lecture Splitting}.

It is important to note that the tutorial and lab workloads are already split into groups of 30-40 students, as a result, do not need to be split further. Thus, as long as a reasonable number of tutorial and lab groups are allocated to each faculty, the tutorial and lab workload distribution remains fair.

% The lecture workload of a course can also be impacted by the number of factors, such as the number of hours of lectures per week, the number of assignments and exams, the number of students enrolled in the course, etc. Thus, it is important to take into account all these factors while determining whether a course needs to be split. For example, a course with 200 students and 1 hours of lectures per week will have a lower workload than a new course with 100 students and 3 hours of lectures per week due to lower preparation time required. Thus, it is important to take into account all these factors while determining whether a course needs to be split.


\subsection{Naive Class-Size Based Splitting}

The workload of a course can be significantly impacted by the number of students enrolled in the course, due to the increased number of assignments and exams to be corrected. For example, all other things being equal, a course has class size of 500 students will have much higher workload than a course with a class size of 50 students. Thus, the class size of a course is a good metric to determine whether a course needs to be split.

The current practice is to split the workload of a course if the number of students enrolled in it exceeds certain thresholds. For example, if more than 200 students enrolled in a course, the course is split between two faculty. The rationale behind this is that in addition to being an indicator of the effort required to teach the course, the number of students is also a limiting factor as the number of halls big enough to accommodate such a large number of students is limited. However, this is not a good strategy as it does not take into account other factors that affect the workload. For example, a course with 200 students and 1 hours of lectures per week will have a lower workload than a new course with 100 students and 3 hours of lectures per week.

However, the workload of a course can be high due to other factors as well. For example, a new course can have a high workload due to the time required to prepare the course material. Thus, it is important to devise a more comprehensive strategy to determine whether a course needs to be split.

\subsection{Workload Based Splitting}

Since the workload is a better indicator of the effort required to teach a course, it is better to directly use the workload as a metric for splitting the course. For example, if the workload of a course exceeds a certain threshold, the course is split between two faculty. However, additional factors need to be incorporated to account for availability of lecture theatres. This led to the development of a 3-Pass splitting strategy that is more inclusive. The 3-Pass splitting strategy is discussed in the next section.

\section{2-Pass Splitting Strategy}

The 3-Pass splitting strategy has passes that take care of individual aspects that require splitting of lectures. The passes are as follows:

\begin{enumerate}
  \item \textbf{Pass 1: Workload distribution perspective}

        This pass ensures that workload distribution among the faculty remains fair and equitable. We split courses that are greater than 300 units between two faculties and courses that are greater than 500 units between three faculties. This ensures that faculty gets reasonable amount of time to prepare for the course and correct assignments and exams.

  \item \textbf{Pass 2: Realizability}

        While the first two passes are pre-emptive, this step is reactive, and done realtime throughout the allocation process. This pass ensures that courses are allocated even in the absence of faculty which have the bandwidth to teach the course. This is done by splitting the course between the faculty that are available. This pass is further discussed in a later chapter
        % TODO: Add reference
\end{enumerate}

\section{Demonstration of the Lecture Splitting Strategy}

% TODO
To be completed.
