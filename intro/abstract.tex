
\chapter*{Abstract}

In a research-intensive university, allocation of teaching workload is one of the most important activities of the academic year. It has wide-ranging implications, affecting the learning outcomes of students, the research output of faculty members, faculty attrition and retention, and the overall reputation of the university. Despite its importance, the process of teaching workload allocation is often manual, with a limited scope and a lack of transparency. This leads to an error-prone process that often results in disproportionate workloads for some faculty members.

First and foremost, the thesis proposes a systematic approach to quantify the research workload of a faculty based on an analysis of the research staff supervised by them. This was found to be more accurate than the existing approaches based on the research output of publications, due to research output being a lagging indicator of research workload. Additional factors like the hierarchical nature of research supervision were accounted for, resulting in a more accurate representation of the research workload of a faculty member.

The service workload of the faculty was simply quantified as a sum of the service duties assigned to them. Using the research and service workload, a new workload model was designed to provide an accurate representation of the faculty workload. This workload model was then used to determine the teaching workload to allocate to each faculty member. The workload model was found to provide a clear representation of the faculty workload, adding much-needed transparency and visibility for the faculty members and the management. The teaching workload derived from the workload model was also found to be fair and equitable.

A comprehensive approach was developed to quantify the workload involved in teaching a lecture. This saliently accounts for different stages of teaching a lecture - preparation, review, delivery, and grading. It is also able to account for the factors that impact the workload - faculty familiarity with the course having an impact on the preparation, and the number of students having a significant impact on the grading workload. The approach was validated using real-world data, and the representation of workload was found to be highly accurate.

Certain lectures were found to have a significantly high workload, which increased the potential for faculty overloading. To alleviate this, a lecture-splitting technique was developed to reduce the impact of very large class sizes, and new courses, which were found to be a major cause of faculty overloading. The technique was designed to split high-workload lectures into two or more smaller lectures, which could then be taught by different faculty members. The technique was found to be highly effective in reducing the workload of faculty members, while also improving the learning experience of students.

Some lectures had an acute shortage of eligible faculty members to teach them, which had the potential to result in unallocated lectures. To resolve this, a pre-emptive allocation of courses was proposed, which involves identifying courses with an acute shortage of faculty supply and allocating them before the main allocation process begins and allocating them to eligible faculty members. This effectively reduced the number of unallocated lectures by 15\%.

To allocate lectures to faculty members, a novel technique was proposed using the Hungarian Algorithm. The allocation process incorporated both the faculty preferences and the teaching performance, to maximize faculty satisfaction without compromising on teaching effectiveness and student learning. Moreover, the proposed technique for lecture allocation also considers the management preference of allocating the best-performing faculty members to earlier-year courses. With this, 63\% of the lectures were allocated.

To resolve the problem of unallocated lectures, three additional techniques were devised which targeted specific scenarios that resulted in unallocated lectures. The first technique was Dynamic Splitting, which involved splitting lectures if they remained unallocated due to faculty workload limits. The second was Targeted Overloading, which allowed targeted overloading of faculty members to ensure that no lectures remained unallocated, which will be countered with tutorial and lab underloading. The third was Dynamic Swapping, which involved swapping the faculty for certain lectures, to allocate other lectures to said faculty. These techniques were found to be highly effective in reducing the number of unallocated lectures and were able to allocate 97.5\% of the lectures.

Finally, an equal misery approach was devised to resolve the problem of unallocated lectures remaining even after the application of the above techniques. This approach involved equitable overloading of the entire faculty cohort, to ensure that no faculty member was disproportionately overloaded. With these techniques applied, the allocation was able to allocate 99.5\% of lectures to faculty members, while ensuring that the priorities of all stakeholders were considered, and faculty workload inequity was avoided.

To allocate tutorials and labs to faculty members, the technique based on the Hungarian Algorithm was extended to incorporate the additional constraints of tutorial and lab allocation. Tutorials and labs face a unique challenge of fragmentation, i.e. the allocation of a tutorial or lab to multiple faculty members, which results in a poor learning experience for students and a high workload for faculty members. To alleviate fragmentation, two techniques were devised which in combination were able to avoid fragmentation. The first technique involved batching tutorials and labs, which involved allocating multiple tutorials or labs to a single faculty at a time. The second technique introduced a consistency bias in the Hungarian Algorithm, which preferentially allocated tutorials and labs to faculty members who were already teaching a course. These techniques were able to reduce the fragmentation from 3.2 faculty members per course to 1.4 while ensuring the teaching effectiveness is not affected. With these techniques, this process was able to allocate 76\% of the tutorials and 75\% of the labs.

To resolve the problem of unallocated tutorials and labs, Dynamic Limit Relaxation was proposed, which involved relaxing the faculty workload limits or limits on the maximum teaching duties for a targeted set of faculty members, to allocate the remaining tutorials and labs. This is done iteratively until all tutorials and labs are allocated and was done in a targeted manner to avoid widespread overloading. This allowed the allocation of 100\% of the tutorials and labs while overloading 9.6\% of the faculty members, of which only 1.8\% were heavily overloaded.


The above methods and techniques were implemented and validated using real-world data, and the results were found to be highly effective, accurately quantifying the workload in a variety of scenarios, and allocating lectures, tutorials, and labs to faculty members while avoiding problems of fragmentation, and faculty overloading, and reducing the need for manual intervention. The allocation was also found to be highly repeatable and deterministic, due to the choice of an algorithmic approach to allocation, which contrasts with the more heuristic-based approaches like Genetic algorithm and Simulated Annealing, which have previously been used to solve the workload allocation problem.


In addition, the allocation techniques were found to be highly scalable, with the ability to perform the faculty workload model and course workload calculations, data preparation, and allocation of 1152 lectures, tutorials, and labs in 30 seconds, making it suited for very-large universities. This speed and scalability were carefully achieved through the choice of algorithm, batching of tutorials and labs, year-wise allocation of lectures, and other steps taken to ensure a highly efficient allocation. Most importantly, the proposed technique ensures a similar proportion of lectures, tutorials, and labs for each faculty member, further ensuring workload equity.

\addcontentsline{toc}{chapter}{\textbf{Abstract}}
