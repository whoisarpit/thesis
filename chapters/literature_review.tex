\chapter{Literature review}


The course allocation problem in large-scale research-intensive universities is a complex problem that has been an under-researched area. This is at odds with the impact of the course allocation problem on the academic community, which can prove to be the limiting factor in the academic performance of students as well as the research outputs of the faculty. While the existing solutions form a good starting point, university management has continually reverted to manual allocation processes, which are time-consuming and error-prone. This chapter aims to provide a brief overview of the course allocation problem, the existing solutions, and the key areas of improvement regarding the course allocation problem.

\section{Introduction}

The course allocation process is a complex process that involves multiple stakeholders, each with their own set of expectations. The key stakeholders are the faculty, the students, and the management. The faculty expects a fair allocation of courses that align with their research interests and teaching preferences. The students expect the best possible teaching experience, with faculty members who have had good feedback in the past. The management expects an optimal utilization of faculty, and to ensure the long-term learning outcomes of the students are met.

The allocation process has various prerequisites that need to be provided for the allocation to be effective. The course allocation process can be divided into multiple different sub-problems, each with its own set of objectives and constraints. These sub-problems are:

\begin{enumerate}
  \item \textbf{Course Workload Modelling}

        The first step in the allocation process is to quantify the workload that is involved in teaching a course. This involves quantifying the time required for each of the activities involved in teaching a course. This is a key step in the allocation process, as it provides a clear picture of the workload involved in teaching a course. This is critical in ensuring that the faculty is not overloaded with courses and that the faculty is not under-utilized.

  \item \textbf{Workload Distribution}

        The next step in the allocation process is to distribute the workload among the faculty. This involves quantifying the non-teaching aspects of their workload like Research and Service, and then distributing the teaching workload among the faculty based on their non-teaching workload. This is a key step in the allocation process, as it ensures that the distribution of the teaching workload is fair and equitable.

  \item \textbf{Allocation}

        The final step in the allocation process is to allocate the courses to the faculty. This involves taking into account the preferences of the faculty and the student feedback and then allocating the courses to the faculty. This also involves taking into account the constraints of the faculty, like the amount of workload they can handle, while ensuring optimal utilization of the faculty. This allocates the lectures, tutorials, and labs to the faculty.

\end{enumerate}

In the following sections, the existing approaches to each of these sub-problems will be investigated. This involves examining the existing models of quantifying the workload involved in teaching a course, research, and other areas of non-teaching workload. Then, an exploration into existing approaches to model this workload is carried out. This is followed by an investigation into the existing approaches to allocating the courses to the faculty, as well as the existing algorithms for allocation problems, and their applicability to the course allocation problem.

\section{Gaps in The Existing Workload Allocation}

In 2009, \textit{Vardi et al.} stated that ``Factors identified by both Heads and focus group members as contributing to workload included lack of sufficient staff, the rise in general administrative duties, inefficient processes, change initiatives, technology, and additional paid work, such as consultancies and offshore teaching.'' \cite{vardi2009impacts}. They also stated that even after the implementation of Workload Allocation Models (WAMs) in a large Australian university, significant numbers of academic staff were still dissatisfied with their long working hours and their workloads.

In a 2014 paper, \textit{Kenny et al.} stated that ``it would be reasonable to expect that mature workload allocation models would be in widespread operation, and there would be a good understanding of the process by academic staff. The researchers became aware, however, of growing anecdotal evidence that many academic staff were feeling overloaded due to the pressures alluded to earlier.'', the pressures being towards greater productivity and increased managerial control over resource allocation \cite{kenny2014effectiveness}. A survey in the same study revealed that only 21\% of the faculty members agreed (20\% unsure) that the workload model reflected the balance and complexity of their work as academics and only 14\% (19\% unsure) agreed that their model was based on realistic estimates of the time required to do the range of tasks they perform as academics.

A systematic review on burnout of university teaching staff by \textit{Watts et al.} \cite{watts2011burnout} identified various factors that contribute to burnout of university teaching staff, including the necessity to teach large volume of students with bad student-staff ratios. In 2017, \textit{Darabi et al.} found that the causes of occupational stress in academic faculty staff included but were not limited to heavy workloads, administrative burdens, and time pressures \cite{darabi2017qualitative}.

\textit{Jensen et al.} identified that staff not being given adequate time to account for the preparation and marking of courses was a key problem unrecognized by the management, which led to undue stress for the faculty members. The staff also indicated that ``they are not averse to hard work, but there is a desire to work effectively and to have a greater measure of preferential time management for the hidden work that improves teaching quality''. A round of interviews revealed that academics at the university found ``the time pressures of workload calculation compromised reflective teaching, undermined their scholarly contribution to knowledge production. Such pressures also affected their capacity to both produce and transmit contemporary, evidence-based knowledge. Research, curriculum development, community engagement, mentoring of colleagues, and the core business of teaching, were all described as compromised by the various attempts to ensure that academics were fulfilling their quota of hours'' \cite{jensen2009vanishing}.

\section{Quantifying Course Workload}

In a case study for the development of a quantifiable academic workload model, \textit{Kenny et al.} state that a lack of clarity about how to quantify academic workload leads to deterioration of academic performance \cite{kenny2012placing}. They also stated that the nature and extent of academic work must be accounted for credibly and transparently and that the allocation of workload should account for these factors.

\textit{Houston et al.} \cite{houston2006academic} argued that the diversity of academic work makes it difficult to arrive at a clear standardized quantification of academic workload. However, several attempts have since been made to quantify the academic workload. \textit{Vardi} \cite{vardi2009impacts} broadly categorizes existing workload models into three categories:

\begin{enumerate}

  \item \textbf{Contact Hours based}

        The contact hours-based models are the least complex of the three model types. Typically, an area will have in mind a target number of contact hours per teaching week in the semester study period. This is a time-based approach used to standardize teaching commitments, relying on typical values without accounting for the nature of the contact, such as lectures, tutorials, or labs, or other variables like class size or course novelty. This model includes standardized additions for tasks like preparation and assessment, and administrators may provide additional support for factors like class size outside of this system.

        \textit{Vardi} stated that ``The shortcomings of this type of model can be its ability to deal with the full range of academic work due to its focus on teaching and administrative duties.'', however, the simplicity of the model was praised, saying ``the Heads using a contact-hours based model were extremely positive about this type of model noting that it was simple to use and easy to understand.'' \cite{vardi2009impacts}

  \item \textbf{Actual Hours based}

        The 'Actual hours' model is a more detailed time-based approach that takes into account the specific nature of a faculty member's activities. These models strive to cover all facets of a faculty's workload by estimating the time required for each task. For example, they consider the type of assessment and the number of students for marking tasks, as well as the time spent in student consultations and responding to emails. This is done to align more closely with human resourcing and costing.

        Regarding this model, \textit{Vardi} stated that ``Over and above the problems noted for all models, both Heads and the actual-hours focus group members were concerned about the reduction of work to a formula and the difficulty in providing true allocations of time in a standard formula when staff work and teach in different ways. The Heads noted that this can lead to arguments about allowances.
        Some Heads were also concerned about some of the staff manipulating the system through, for example, double dipping, when putting in their claims for allocations.'' \cite{vardi2009impacts}

  \item \textbf{Points based}

        The 'Points model' is an alternative method for quantifying workload, using a system that converts various faculty activities into 'points', loosely based on the hours of effort required. Faculty members accrue points for different tasks which provides more flexibility, but \textit{Vardi} stated that ``this type of model garnered the most negative response from both Heads of areas and the points-based focus group members alike. In addition to the problems that beset all the models, both the Heads and focus group members overwhelmingly expressed concerns about the complexity of this type of model and the lack of relationship between points and hours.'' \cite{vardi2009impacts}
\end{enumerate}


\textit{Vardi} also stated that while a simple model is highly desirable for ease of use, it is important to ensure that the model is not too simple to be ineffective. The model has to account for the variance and complexity of the academic workload among different faculty. These competing priorities of simplicity and comprehensive workload quantification are key in the design of a workload allocation model \cite{vardi2009impacts}. This study was conducted in 2009, and since then, several new models have been proposed.

In 2020, \textit{Griffith et al.} \cite{griffith2020framework} proposed a framework for quantifying the course workload which divides the teaching workload into three parts - classroom workload ($T_c$), administrative workload ($T_a$), and preparation workload ($T_p$). ($T_c$) is calculated by the total number of hours spent in the classroom and labs per week multiplied by the number of weeks that teaching is required during the contracted period. ($T_a$) is approximated by multiplying the total number of semester credits taught as prescribed by payroll purposes during the contracted period by the 16 weeks in a typical semester. ($T_p$) is estimated by the number of unique courses taught in a contracted period multiplied by four weekly hours and then multiplied by the number of semester credit hours assigned to each course. \textit{Griffith et al.}'s model illustrated the negative effects that heavy teaching loads and unique course preparations have on a faculty member's contracted time for research and service.

\textit{Jensen et al.} mentioned the lack of time being provided for marking of courses, and preparation of courses. This led to faculty investing their own time to mark courses and felt under-recognized for the time and effort they put into the preparation of contemporary, quality course material. This is a key issue with the contact hour-based approach, as it fails to account for the time spent on these activities \cite{jensen2009vanishing}.

\textit{Kenny et al.} found that the actual hours approach needed to accurately quantify the time spent on a plethora of activities in the form of hours, some of which are not directly measurable. This led to widespread dissatisfaction among faculty who had a lack of transparency into the time allocations for the activities and felt that a better solution was needed. This leads to ambiguity and a case-by-case approach to workload allocation, which is not scalable and leads to a lack of transparency \cite{kenny2014effectiveness}.

In a 2021 study on the emerging state of workload allocation, \textit{Kenny et al.} identify that the workload allocation's inability to identify key tasks involved in teaching, such as preparation and marking of courses was stated as the primary challenge towards a fair workload allocation. This emphasizes a need for a more comprehensive quantification of course workload that accounts for all the activities involved in teaching \cite{kenny2021emerging}.

\section{Quantifying Non-Teaching Workload}

The working life of a university faculty comprises primarily three parts - Service, Research, and Teaching. Each university has a different set of expectations from the faculty in each of these areas, but broadly speaking, the contributions towards furthering the university's research outputs are considered as Research Workload, and all other workloads except Teaching and Research are considered as Service Workload. This includes administrative duties, community service, and other activities that are not directly related to teaching or research.

Regarding the faculty view on different types of workload, \textit{Finlay et al.} stated that ``The popular view amongst academics is that supervision is satisfying but demanding; that
teaching is the most enjoyable activity until it becomes overwhelming, and that the associated marking rapidly becomes tedious; that administration is a chore; that research is the supreme activity if only the time were available. In practice, few academics have the talent or the motivation to excel in all four components. The university system provides incentives that, for example, tend to emphasize excellence in research more than in the other components. The fact that the incentive of promotion then results in a greater administrative load is a peculiar contradiction in the system.'' \cite{finlay1994management}.

Additionally, \textit{Deem et al.} identified that placing pressures on academic staff to increase their number of publications is viewed unfavorably, as it leads to a focus on quantity over quality. This is a key issue with the research workload, as it is difficult to quantify. Moreover, it is seen that research-active faculty ends up doing more research at the cost of overloading and diminishing the quality of teaching \cite{deem2020new}.

\textit{Sood} proposed a novel approach to quantifying the research workload, by treating the number of research staff supervised by a faculty as a proxy for the research workload. The post-doctoral research staff, in this case, were treated to have half the impact as that of research assistants and research engineers, alluding to the fact that they require less supervision. This approach was found to be effective in practice, as it was able to quantify the research workload without relying on the research output, which is a retrospective measure of the research workload. They also proposed using research and service workloads by directly relaxing the teaching workload expectations from a faculty. \cite{rohan2017}.

\section{Workload Models}

In a 1998 study, \textit{Doyle and Hind} reported limited time for teaching preparation and too heavy a workload as top stressors in a sample of UK academics \cite{doyle1998occupational}. They also cited that nearly half of the faculty members rate either teaching or a combination of teaching and research as their top priorities, but only around 30\% of them believe that their departments share this view.

In 2003, \textit{Winefield et al.}, in an analysis of occupational stress in Australian university staff, reported work overload, poor management practices, and insufficient recognition and reward among the top stressors of academic staff. They also noted that there was high pressure towards publishing, with academics who rarely publish, aren't cited often, or don't attract research funding being viewed as less valuable to the university \cite{winefield2003occupational}.

As part of an ongoing study exploring the lived experiences of Australian academics, \textit{Kenny et al.} identified the emergent challenges posted towards fair allocation of academic workload being - Underestimation or omission of key tasks like marking, preparation of courses, and research, reported by 27.4\% of the respondents, and the lack of transparency in the workload allocation process, reported by 15.3\% of the respondents, and inequities i.e. the need to deal with individual differences in workload, reported by 14.9\% of the respondents \cite{kenny2021emerging}.

As a way to solve these issues, there exists a wide variety of workload models, which aim to give a clear picture of the overall workload distribution among the faculty. These can then be used by the management to gain an understanding of the workload distribution among the faculty to identify areas of workload inequity and also to identify areas of under-supply. They also form a basis for workload allocation, allowing the management to allocate the workload among the faculty fairly and equitably.

In 1994, the STAR Model (Service, Teaching, Administration, Research) proposed by \textit{Finlay et al.} was a key stepping stone in defining the non-teaching parts of a faculty's workload. In this model, a score or classification was devised for each component which could then be reduced to an equivalent rating scale. For Service ($S$) and Administration ($A$), a list of all possible tasks was made, involved in each of these areas, and then rated on a scale of 0 to 3 in steps of 0.1. For research, the rating was based on the output of recent publications, particularly of papers in international journals of standing. A scale of 1 to 4 was then chosen for this component, in increments of 1, rating the faculty from no research output in recent years, to above-average output. The teaching rating was then found by subtracting these ratings from a preassigned constant $M$, which carried a value of 10 in the original model \cite{finlay1994management}.

\textit{Burgess et al.} found that a simpler TRO - Teaching, Research, and Other model was preferred by faculty because of its contemporary nature and official sanction in the Transparency Review. They also stated that transparency in workload allocation is critical to ensure wider acceptance of the workload model by the faculty members \cite{burgess2003academic}.


\begin{equation}
  T_c + T_a + T_p + S_i + S_p + S_c+R = C
  \label{griffith_wam}
\end{equation}

Recently, \textit{Griffith et al.} goes much further in describing the constituents of Service Workload as \textbf{Professional}, and \textbf{Community} Service Workload (\(S_p, S_c\)), and the constituents of Teaching workload as \textbf{Classroom}, \textbf{Administrative} and \textbf{Preparation} Workload ($T_c, T_a, T_p$) as shown in \autoref{griffith_wam}. This provides much-needed clarity into the nature of the work involved in each of these constituents. It also highlights key problems the academic community faces in each of the workload constituents \cite{griffith2020framework}.

In a 2021 journal, \textit{Narasimhan et al.} explored a new TRASE model, accounting for Teaching, Research, Administration, Service, and Entrepreneurship. This provided a workload factor for each of the constituents, given for a level of participation and workload in each of the constituents. ``if a person wants to have a teaching load of 30\%, research load of 50\% and service load of 20\%, then s/he has to have: i) two subjects per semester, ii) three conference papers or 1-3 journal papers per year with one paper in Tier-I level and obtain funding to the tune of 4,00,000/year and have two Ph.D./MS students and iii) participate in 2 committees and perform at least one other significant activity'' \cite{narasimhan32trase}. They also introduce a concept called Agreement To Perform (ATP), where the faculty members set their own goals at the start of the term, which allows them to account for personal and circumstantial factors, like their plans to take a sabbatical. This gives the university administration some visibility into the expected performance of the faculty, reducing clashes between the faculty and the administration.

\subsection{Workload Inequity in Teaching Allocation}

\textit{Jensen et al.} \cite{jensen2009overload} and \textit{Kenny et al.} identify that workload modeling and distribution fails to make an impact due to the lack of additional resources made available to address workload inequities. The lack of available expertise in key areas disproportionately overloads certain faculty with said expertise, essentially punishing faculty for being skilled \cite{jensen2009overload, kenny2014effectiveness}.

\textit{Vardi} and \textit{Houston et al.} \cite{vardi2009impacts, houston2006academic} clearly identify that even with the mandated use of WAMs in Australian universities, overloading of faculty is a widespread problem due to under-supply issues. \textit{Kenny et al.}  also notes that defining realistic time limits is a necessary step towards ensuring welfare of academic staff \cite{kenny2014effectiveness}. One of the big reasons for this was identified as budgetary concerns, with universities being unwilling to hire additional faculty to address the workload inequity \cite{kenny2012placing}.

\textit{Griffith et al.} noted that ``We perceive that institutions set faculty up for failure when the faculty are assigned more than two simultaneous unique course preparations or heavy teaching loads (which we define as in-excess of the equivalent of 12 undergraduate credits per traditional semester) and those faculties are expected to maintain their research productivity''. They also stated ``We propose that institutions shift their perspective on faculty employment contracts, which should be viewed as an agreement with the faculty by the institution to acquire a maximum number of working hours per contract year. An employment contract should not be viewed or treated as an all-access pass to control every hour of each faculty member's time year-round'' \cite{griffith2020framework}.

\textit{Kenny et al.} stated that ``In an under-funded corporate university environment, pressures on academics to produce more have led to a reduction in the self-management aspects of their work. Increasing requirements to meet performance expectations such as outcome-based research targets for publications competitive grant income, and higher teaching loads due to a larger and more diverse student cohort has resulted in increased work intensity for academics. This has raised the need to be able to monitor the competing work demands on academics to avoid overload on individual staff and provide a safe workplace.'' \cite{kenny2014effectiveness}

\section{Allocation Priorities}

\textit{Harwood et al.} identifies the three key stakeholders of workload allocation - the faculty, the students, and the administration. The faculty expects a fair allocation of courses that align with their research interests. The students expect a variety of courses that allow them to customize their academic journey, and expect the best possible teaching experience, with faculty that has had good feedback in the past. The administration expects an optimal utilization of faculty but is largely in alignment with the faculty and student expectations \cite{harwood1975optimizing}.

\textit{Schniederjans et al.} and \textit{Badri et al.} \cite{schniederjans1987goal, badri1998multi} further go into the role faculty preferences play in this distribution, with the faculty being able to express their preferences for the courses they want to teach.

\textit{Sood} \cite{rohan2017} identified another management priority - the need to ensure that the best faculty are allocated to earlier-year courses, which are critical in setting the foundation for the academic journey of students. This was identified as a key priority, as it ensures that the students get the right amount of guidance in the early years, while they gain more autonomy and independence in the later years, and build the skills to self-learn. Additionally, a bias was identified towards allocating faculty to courses they have taught in the past, as they are familiar with the course material and the teaching style, and thus require less work to prepare for the course.

With this in mind, the following allocation priorities were identified:

\begin{enumerate}
  \item Faculty

        — Faculty course preferences are important as they allow them to align their teaching with their areas of research interest

        — Faculty should be allocated to courses that they have taught in the past, as they are familiar with the course material and the teaching style, and thus require less work to prepare for the course

        — Schedule preferences also allow them to create windows in their schedule for focused research and administrative work that fits their work styles

        — Faculty should not be overloaded with courses, as it leads to a lack of time for research, which is important for their career progression

  \item Student Body

        — Students expect a minimum variety of courses that allows them to customize their academic journey

        — Students expect the best possible teaching experience, with faculty members that have good feedback in the past

  \item Administration

        — Maintain equitable distribution of work among the staff and lack of bias

        — Ensure that the best faculty are allocated to earlier-year courses, which are critical in setting the foundation for the academic journey of students.

        — Ensure optimal utilization of faculty

        — Ensure that the faculty members are not overloaded with courses, as it leads to burnout and attrition

\end{enumerate}

\section{Existing Approaches to Course Allocation}

\textit{Schniederjans et al.} \cite{schniederjans1987goal} originally proposed a goal-programming algorithm for solving a preference-aware course allocation algorithm. This model was presented as ``an ideal candidate for a decision support system (DSS). A DSS that could reduce the repetitive faculty assignment problem from a few hours to a few minutes and could be structured so no prior information or goal programming is necessary.''

In 2017, \textit{Sood} \cite{rohan2017} explored a greedy solution because it ``offers multiple advantages, especially over other alternatives like Genetic Algorithms and Linear programming. It is simple to implement, and as a result simple to explain in an organizational setting like a university''. The greedy solution involved iterating through the courses and allocating the best faculty available. The solutions produced by this approach were observed to be satisfactory but tended to converge local minima. This was countered using an Iterated Greedy algorithm inspired by the \cite{ruiz2007simple}, which goes through multiple phases of random destruction and reconstruction of the solution to converge towards a better solution.

\textit{Sood} also explored the application of the Kuhn-Munkres algorithm, commonly known as the Hungarian algorithm, to allocating Student Final-Year Projects. However, this wasn't explored for the course allocation problem \cite{rohan2017}. The Hungarian Algorithm \cite{munkres1957algorithms} is the go-to algorithm for one-to-one matrix-optimization problems, but it isn't built to directly apply to many-to-many solutions. Hence, additional steps need to be taken to orchestrate the many-to-many allocation problem into a one-to-one matrix optimization problem.

In 2021, \textit{Mallicka et al.} \cite{mallicka2021claps} explored the application of the Hungarian algorithm to build a ``Course and Lecture Assignment Problem Solver''. For this, the considerations include faculty expertise, effectiveness in course delivery, and the goal of minimizing preparation time while maximizing teaching effectiveness. It was limited to a one-to-one allocation of faculty to courses and didn't explore many-to-one allocation problems where multiple courses may be allocated to one faculty. It also relied on a self-rated effectiveness score. The study demonstrates the usefulness of the Hungarian method in solving the course assignment problem in an institution.

\section{Algorithms for General Allocation Problems}

The course allocation problem is a many-to-one allocation problem, where multiple courses are allocated to one faculty. Several studies explore various algorithms specifically built for many-to-one allocation problems and a few other studies explore the applicability of one-to-one allocation algorithms to many-to-one allocation problems.

Recently, \textit{Zhu et al.} \cite{zhu2016solving} proposed a new variant of the Hungarian algorithm, called the Kuhn-Munkres with Backtracking algorithm ($KM_B$), which is a many-to-many variant of the Hungarian algorithm. It maintains the worst time complexity of the Hungarian algorithm, while also being able to solve many-to-many allocation problems. It showed the algorithm to be effective in solving the many-to-many allocation problem, and that it meets the number of successful assignments in comparison to the Exhaustion algorithm in a significantly lesser time. The cost-reduction aspect of the Hungarian algorithm was not evaluated in this study.

Additionally, other algorithms have been proposed for solving the many-to-many allocation problem, like Simulated Annealing \cite{bertsimas1993simulated}, Tabu Search \cite{glover1990tabu}, Stochastic Hill Climbing \cite{juels1995stochastic}, Memetic Algorithms \cite{neri2012memetic} and Genetic Algorithm \cite{lambora2019genetic}, which are effective in solving the many-to-many allocation problem. The following is a brief overview of each of these algorithms, along with their pros and cons. The list also mentions if the output is deterministic, i.e. if the same input is given to the algorithm, it will always produce the same output, or if it is probabilistic, i.e. if the same input is given to the algorithm, it may produce different outputs.

\begin{enumerate}
  \item \textbf{Genetic Algorithm}

        This is a search heuristic that mimics the process of natural evolution. It uses methods such as mutation and crossover to generate new offspring. It's faster than other algorithms and can provide multiple solutions quickly. It requires less information about the problem. The random heuristics sometimes don't find the optimum. It is also not a complete algorithm and can be computationally expensive. The allocation generated by it is probabilistic \cite{lambora2019genetic}.

  \item \textbf{Stochastic Hill Climbing}

        It's an optimization algorithm that makes use of randomness as part of the search process. It's appropriate for non-linear objective functions where other algorithms like gradient descent fail. It's suitable for a variety of optimization problems such as scheduling, route planning, and resource allocation. It has limitations such as the tendency to get stuck in local maxima and the lack of diversity in the search space. The allocation generated by it is probabilistic \cite{juels1995stochastic}.

  \item \textbf{Simulated Annealing}

        This is a probabilistic technique used for finding an approximate solution to an optimization problem. It's inspired by the annealing process in metallurgy. It's efficient, easy to implement, and theoretically sound. It suffers from a slow convergence rate. The allocation generated by it is probabilistic \cite{bertsimas1993simulated}.

  \item \textbf{Tabu Search}

        This is a metaheuristic search method that guides a local heuristic search procedure to explore the solution space beyond local optimality. It's a general tool that can be adapted to a wide range of problems. The allocation generated by it is mostly probabilistic \cite{glover1990tabu}.

  \item \textbf{Memetic Algorithm}

        This is an extension of the genetic algorithm. It uses a local search technique to reduce the likelihood of premature convergence. It can solve complex optimization problems effectively. The allocation generated by it is probabilistic \cite{neri2012memetic}.

  \item \textbf{Hungarian Algorithm}

        Also known as the Kuhn-Munkres algorithm, it's used to find maximum-weight matchings in bipartite graphs, which is sometimes called the assignment problem.
        It's an exact algorithm with a worst-case time complexity of $O(n^3)$, where n is the number of vertices. It's not efficient computationally in some cases. It is tough to apply for problems with many interdependencies. The allocation generated by it is deterministic \cite{munkres1957algorithms}.
\end{enumerate}

Additionally, \textit{Dofadar et al.} have recently applied a hybrid approach which was a combination of the Local Repair Algorithm and the Modified Genetic Algorithm, showing a higher accuracy in solving the many-to-many allocation problem, than other existing solutions. This was also shown to be faster than other stochastic solutions \cite{dofadar2021hybrid}.

\section{Summary}

% Teaching Workload

\textit{Griffith et al.} identified the constituents of the teaching workload - Classroom, Administrative, and Preparation Workload, and provided an approach towards measuring each of these constituents. They also identified unique course preparations as an important factor that impacts the workload \cite{griffith2020framework}. \textit{Lowenthal et al.} also identified the impact of class size on workload. However, none of the models incorporated class size and faculty familiarity with the course or differentiated between different types of teaching activities - lectures, tutorials, and labs. Thus, a more comprehensive approach towards quantifying the teaching workload, which accounts for the impact of class size, the type of teaching activity, and the familiarity of the faculty with the course, may be more effective in quantifying the workload involved in teaching a course.

% Splitting

Additionally, \textit{Watts et al.} identified the necessity to teach a large volume of students with bad student-staff ratios, as a key factor in burnout of university teaching staff \cite{watts2011burnout}. Some universities have a policy of allocating multiple faculty to split the workload of a large course, which has proven to be effective in reducing the workload of the faculty. This is an area that needs to be explored further.

% Research and Service Workload

\textit{Finlay et al.} proposed a simple model to quantify research workload based on the quantity and importance of the research output \cite{finlay1994management}. However, research output as a measure of research workload was found to be ineffective due to high variability in the work required to produce research output. Research output was also found to be a lagging indicator of the research workload. \textit{Sood}'s approach of using the number of research staff supervised by a faculty to measure research workload proved effective in this regard \cite{rohan2017}. However, more research is required to account for the economies of scale achieved by the hierarchical nature of research supervision. The approach towards quantifying service workload proposed by \textit{Finlay}, \textit{Sood} and others was found to be effective \cite{finlay1994management, rohan2017}.

% Workload Models

The STAR model proposed by \textit{Finlay et al.} \cite{finlay1994management} and the TRASE model proposed by \textit{Narasimhan et al.} \cite{narasimhan32trase} provided a clear picture of the various constituents of the faculty workload and the factors that impact each of these constituents. These were a key step in the right direction, as they provided high visibility into the faculty workload. They can also be used to derive an equitable workload allocation for the faculty. A new model that takes the comprehensiveness of these models while also accounting for updated methods of quantifying the teaching, research, and service workloads, may provide much-needed clarity to the management, and help in identifying areas of workload inequity and under-supply.


% Lecture Allocation

The approaches to allocation opted by \textit{Schniederjans} and \textit{Sood} establish a good baseline for the allocation process. However, each suffers from problems of local optima and provides limited flexibility. The recent advancement by \textit{Dofadar et al.} in applying a hybrid approach which was a combination of the Local Repair Algorithm and the Modified Genetic Algorithm, showed very promising results, which highlighted the possibility of exploring newer algorithms for lecture allocation \cite{schniederjans1987goal, rohan2017, dofadar2021hybrid}.

% General Allocation

Several recent algorithms have been used to solve general many-to-many allocation problems, showing promising results. The modified Kuhn-Munkres algorithm with Backtracking, proposed by \textit{Zhu et al.}, was also found to be effective and can be used to expand upon the research done by \textit{Malliacka et al.} in applying the Hungarian Algorithm to the problem of course allocation \cite{dofadar2021hybrid, zhu2016solving, mallicka2021claps}. Hungarian algorithm is also preferred over stochastic approaches due to its deterministic nature, which is critical in the course allocation process, Thus, further exploration is needed in applying newer approaches, especially the Hungarian algorithm, to the course allocation problem.

Overall, existing research on the course and workload allocation problems forms a good baseline for the allocation process. There is scope for improvement in the quantification of the faculty workload, especially in the teaching and research workload. Additionally, existing workload models showed promising results, and combining these with the newer approaches to quantifying the faculty workload may provide transparency to the management. Additionally, the application of newer algorithms to the course allocation problem may provide a more effective solution to the course allocation problem. The following chapters aim to build upon the existing research to provide a more effective solution to the course allocation problem.
