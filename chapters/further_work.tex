\chapter{Conclusion and Future Work}

\section{Conclusion}

In this thesis, the problem of automated teaching workload allocation to faculty members was addressed. The proposed solutions were found to be holistic and effective and were able to allocate all teaching workload to the faculty, while ensuring that the allocation was fair and equitable, and prioritized the student feedback, the faculty preferences, and the management priorities.

To allocate the courses, the problem was split into four sub-problems - determining the workload of each faculty in the domains of research and service to equitably distribute the workload, determining the workload of the teaching activities of the courses, allocating lectures of a course to the faculty, and allocating tutorials and labs of a course to the faculty.

First, this thesis proposed an approach to quantify the research workload of the faculty, which was achieved based on the management of the research team under the faculty's supervision. There was a non-linear correlation found between the size of the research team and the research workload of the faculty, owing to a naturally hierarchical structure of staff management, with some staff being directly managed by the faculty, who in turn manage the rest of the research staff, especially for larger research teams. This non-linearity was modeled using a hyperbolic tangent function, and the approach was found to be effective in quantifying the research workload of the faculty.

The service workload was measured based on the individual service activities which were manually rated by the management, which were added to derive the service workload of the faculty. The workload modeling into the RTS ratio and it's subsequent use in allocating teaching workload to faculty was found to be effective in ensuring that the faculty workload was equitably distributed, and the faculty were not overloaded with work.

Next, the approach for quantifying the teaching activities of a course involved diving the activity into four distinct constituents, namely - preparation, review, delivery, and grading workload. The workload for each of these constituents was modeled individually for lectures, tutorials and courses, accounting for the effects of class size and faculty familiarity with the course. This showed that the workload of lectures was anywhere from 150 to 700 units, being highly dependent on the class size and faculty familiarity, while the workload of tutorials and labs was more uniform, at 20-40 units, with little impact from class size and faculty familiarity.

The high workload variance of lectures was found to be difficult to allocate due to high granularity, with the workload of a few lectures being higher than the workload limits of any faculty. To address this, a splitting strategy was introduced, which split the workload of a lecture into 2-3 smaller lectures, reducing the granularity and allowing for easier allocation of these lectures.

An allocation technique, based on the Hungarian Algorithm, was proposed for the automated allocation of lectures to faculty members. It allocated lectures based on faculty preference, student feedback, and the management priority of preferential allocation of earlier-year courses to the best faculty. The preferences of the faculty were honored, with $>$50\% of the faculty teaching lectures of their top 5 preferences. However, only 47\% of the lectures were allocated. This was found to be due to multiple factors.

One of the issues was related to the management's priority for preferential allocation of the best faculty to earlier-year courses, which resulted in a shortage of faculty to teach the later-year courses. This was clearly seen with only 11\% of the Year 5 courses being allocated. However, since this priority was non-negotiable, the allocation process was adapted to account for this by pre-emptively allocating latter-year courses with a shortage of faculty supply. This was able to allocate an additional 16.3\% of the lectures to the faculty with an asymetrical improvement in latter-year courses, with 89.1\% of the Year 5 courses being allocated.

The other issues were primarily related to demand and supply mismatches, with the availability of faculty and expertise for certain courses being limited. Firstly, the coarse granularity of lectures, which was somewhat resolved by the initial splitting strategy, still resulted in a few unallocated lectures. To resolve this a dynamic splitting strategy was introduced, which recognized lectures that weren't allocated due to their workload being too high. It checked if the allocation of these lectures is possible by splitting, and if so, split the lecture. This was able to independently allocate an additional 22.5\% of the lectures to the faculty.

It was found that some lectures couldn't be allocated even with splitting, due to a shortage of faculty supply. To alleviate this, targeted overloading of faculty was introduced, with the foresight that the additional workload would be compensated by underloading the faculty in the subsequent allocation of tutorials and labs. This was able to independently allocate an additional 35.8\% of the lectures to the faculty.

Finally, a dynamic swapping strategy was introduced, which retrospectively identified the lectures that couldn't be allocated due aforementioned prioritization of earlier-year courses, and attempted to free up faculty members for teaching these lectures, by swapping their lectures with other faculty while ensuring that the teaching performance was not compromised. This was able to independently allocate an additional 15.6\% of the lectures to the faculty.

These techniques were independently analyzed to avoid any confusion on the causality of the improvement. When combined, these techniques were able to allocate 97.5\% of the lectures to the faculty members. Finally, it was also found that, in certain cases, the impact of targeted overloading was muted, due to the lack of foresight of the earlier step in realizing that overloading would be necessitated. To address this, a pre-emptive overloading of all the faculty members was introduced if the aforementioned techniques were unable to allocate all the lectures to the faculty. This was able to allocate 99.4\% of the lectures to the faculty members. An allocation of all the lectures would be achieved if the workload limits were relaxed for the remaining 0.6\% of the lectures.

To allocate tutorials and labs to faculty members, the proposed techniques based on the Hungarian Algorithm were adapted to account for additional challenges unique to tutorials and lab sessions. It was found that a problem of fragmentation exists, with the tutorials of many courses being allocated to each faculty, which would result in additional overhead. To address this, a consistency bias was introduced to favor the faculty members who were already teaching other parts of the same course in some fashion. This was found to be effective, reducing the number of faculty teaching the tutorials and labs of the same course from 4.6 faculty teaching tutorials and labs of a course to 3.8 faculty. However, it was noticed that modifying the cost of allocation to bias the Hungarian Algorithm towards a certain allocation led to marginal improvement, and a more fundamental change in the approach was needed for further improvements.

To address this, a batching technique was introduced, which allocated tutorials and labs in batches of 6 sessions at a time, reducing fragmentation. This batch size was found to be suitable for the current dataset but would need to be adapted for other datasets. The batching was able to further reduce the number of faculty teaching the tutorials and labs of the same course to 1.9 faculty, which was a significant improvement over the original 4.6 faculty. It is also important to note that both of these techniques were able to achieve this improvement without adversely affecting workload equity and quality of allocation.


Certain unallocated tutorials and labs remained, which were again found to be due to supply-demand asymmetries. The pre-emptive allocation of tutorials was introduced, which was able to reduce the number of unallocated tutorials by 3\%. Additionally, a selective workload relaxation technique was introduced, which identified and iteratively relaxed the workload limits causing the unallocated tutorials and labs to allow their allocation. An iterative increase was preferred to ensure that workload fairness was not drastically compromised. This reduced the number of unallocated tutorials and labs from 27\% to 0\% without adversely affecting workload equity.

The tutorial and lab allocation process, in combination with these techniques, was able to allocate 100\% of the tutorials and labs to the faculty members in an automated manner, at the expense of overloading 9.6\% of the faculty members, of which only 1.8\% were heavily overloaded.

Unlike existing heuristic-based approaches, the proposed techniques were also found to be highly deterministic and scalable. The batching of tutorials further helped achieve this scalability, due to the reduction in matrix sizes and iterations for the Hungarian Algorithm. The entire allocation process allocated the activities of an entire semester involving 1152 lectures, tutorials, and labs, within 30 seconds on a standard PC, and was also able to adapt to the various challenges that may be posed by a variety of datasets.

\section{Future Work}

The allocation of teaching workload to faculty is a complex problem, and while the allocation achieved in this thesis is fair and equitable, there are still certain areas of improvement that can be made. Some of these areas are discussed below.

\subsection{Inter-semester workload balancing}

It was seen that certain faculty members needed to be overloaded in a semester to ensure the allocation of all teaching workload to the faculty. There is scope for inter-semester workload balancing, where the workload of the faculty is balanced across semesters, to ensure that in the event of an overloading of faculty in one semester, the workload of the faculty is reduced in the subsequent semester.

\subsection{Workforce Gap Analysis}

Techniques like pre-allocation highlighted the shortage of eligible faculty for certain courses. This can be used to perform a workforce gap analysis, which can be used to determine the number of additional faculty that need to be hired to ensure that all courses can be taught by eligible faculty. A closer examination of the degree of deviation from faculty preference should pave the way for identifying a shortage of expertise in certain areas. This, in combination with the need for systematic overloading of faculty, shows the possibility for a more targeted hiring process, which can be beneficial for the institution.


\subsection{Realization of Institutional Goals}

The RTS model provided a transparent and insightful metric for the workload of the faculty. This can be used to determine if the institutional goals are being met, by ensuring that the aggregate RTS ratio of the institution is in the right direction. A research-intensive university may aim for an aggregate RTS Ratio of $R:T:S = 6:4:2$, for example. The RTS ratio can also be used to determine if these goals are being met at the department level, and faculty level, and can be used to identify areas of improvement.

\subsection{Accounting for the impact of Teaching Assistants}

Teaching Assistants (TAs) are typically graduate students who assist the faculty in teaching the course. Their role is to assist the faculty in teaching activities, such as grading, and in some cases, they may also be involved in the delivery of the tutorials and labs. Their presence might help in reducing the workload of the faculty, and can also be used to improve the learning outcomes of the students by providing them with additional support. The impact of TAs on the teaching workload of the faculty is not accounted and thus, adapting the proposed models for TA supply and involvement would be a worthy pursuit.

\subsection{Dynamic Batching of Tutorials and Labs}

With the static batching of tutorials and labs, the batch size needed to be tweaked to achieve a balance of granularity and fragmentation, based on the dataset. An improvement to this would be a dynamic batching strategy, which takes a more nuanced look at batching decisions for individual courses based on faculty availability. This would hypothetically be able to achieve the best balance of granularity and fragmentation, and would also further reduce the need for any manual intervention. Thus, a dynamic batching strategy would be a worthy pursuit.

\subsection{Timetabling Constraints}

Timetabling of the courses is carried out separately, independent of the allocation of teaching workload to faculty. However, the two processes are interdependent, and if two courses are timetabled at the same time, they cannot be taught by the same faculty. This can be incorporated into the allocation process, to ensure that the courses are timetabled in a way that allows the courses to be taught by the same faculty.

Other timetabling constraints like the distance between classrooms can also be incorporated into the allocation process, to ensure that faculty have adequate time to reach the classroom after teaching another class. This can be explored in future work.
