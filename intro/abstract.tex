
\chapter*{Abstract}

In a research-intensive university, fair and equitable allocation of teaching workload is one of the most important activities of the academic year. It has wide-ranging implications, affecting the learning outcomes of students, the research output of faculty members, faculty attrition and retention, and the overall reputation of the university. Despite its importance, the process of teaching workload allocation is often manual, with a limited scope and a lack of transparency. This leads to an error-prone process that often results in disproportionate workloads for some faculty members.

In this research, a detailed exploration of the existing approaches toward the improvement of teaching workload allocation was conducted, identifying the strengths and challenges associated with the approach. One key challenge was the lack of a comprehensive model to quantify the workload involved in the teaching activities of a course, which takes into account the various factors that impact the workload.

This led to the development of a comprehensive approach for the time-based quantification of the teaching workload involved in the various stages of teaching a course, like preparation, delivery, and assessment. The model was designed to be flexible, allowing for the inclusion of various factors that impact the workload, such as course newness, class size, and activity type. A lecture-splitting algorithm also was developed to reduce the impact of high-workload lectures, which was found to be a major cause of faculty overloading. The algorithm was designed to split high-workload lectures into smaller lectures, which were then allocated to faculty members. The model and lecture-splitting algorithm were then validated using real-world data, and the results were found to be highly effective, accurately quantifying the workload in a variety of scenarios and alleviating the problem of faculty overloading.

The workload of faculty members was found to have varying proportions of research and service duties. In the process of equitably allocating the teaching workload to account for this variation, a new workload model was developed, which was called the RTS ratio. The RTS ratio was designed to provide a transparent and accurate representation of the workload of faculty members, taking into account their research activities, service duties, and teaching workload. The RTS ratio was then used to determine the teaching workload that should be allocated to each faculty member, ensuring that the workload was fair and equitable.

The allocation of lectures involved taking into account student feedback, faculty preferences, and the RTS ratio of faculty members, among other factors. The allocation was performed using a novel approach that was based on the Hungarian Algorithm, which was found to be highly effective in allocating lectures to faculty members. Several additional techniques were also developed to ensure the allocation of all lectures and alleviate faculty overloading, reducing the need for manual intervention. The result was a highly effective approach that was able to allocate 99.5\% of lectures to faculty members while ensuring that the priorities of all stakeholders were taken into account.

The allocation of tutorials and labs involved unique challenges, such as the need to avoid fragmentation i.e. the allocation of a tutorial or lab to multiple faculty members, which results in a poor learning experience for students and a high workload for faculty members. The allocation process is built upon the approach used for lectures, using the Hungarian algorithm to allocate tutorials and labs to faculty members. Additional measures were also taken to ensure that fragmentation was avoided, which resulted in a reduction of the number of faculty members teaching a course from 3.2 to 1.4 for labs and from 2.4 to 1.2 for tutorials. The result was a highly effective approach that was able to allocate 100\% of the tutorials and labs to faculty members while avoiding problems of fragmentation, faculty overloading, and reducing the need for manual intervention.

\addcontentsline{toc}{chapter}{\textbf{Abstract}}
