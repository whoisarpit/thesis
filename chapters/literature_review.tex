\chapter{Literature review}

Allocation of teaching activities consists of multiple stages like estimation of faculty's available workload, defining a faculty's appropriateness to teach individual courses, and using algorithms to allocate the teaching activities. This section lists and critically analyses the existing work related to each of these, and investigates the problems in said work, if any, and to find solutions/methodologies to overcome the same.

\section{Workload Distribution}

When it comes to teaching allocation, the first question that arises is how much teaching workload should be given to each faculty. The working life of a university faculty comprises primarily three parts – Service, Research, and Teaching, so the simplest solution to equal distribution of workload between faculties (or equal misery) in all three areas \cite{gray1989university}. However, as different faculties do not display an equal affinity or competency towards these areas \cite{finlay1994management}, equal distribution would lead to inefficient use of the faculties' skill-sets. Moreover, it is seen that research-active faculty ends up doing more research at the cost of overloading and diminishing the quality of teaching. Thus, an equitable approach to workload distribution will account for the amount of work done by faculty in research and service areas to distribute teaching work. Several Workload Allocation Models (WAMs) have been developed that facilitate in quantifying said work.

\section{Existing Workload Allocation Models}

\cite{vardi2009impacts} broadly categorizes existing WAMs into three categories:

\begin{enumerate}

    \item \textbf{Contact Hours based}

          These models focus on only the time required for course-related activities (lectures, tutorials, and labs). Fixed concessions may be given for related activities (preparation of assignments and lectures, assessment etc.), but they fail to account for the additional factors like class size, faculty's familiarity to the course etc. These models entirely ignore non-teaching activities of the faculty workload.

    \item \textbf{Actual Hours based}

          These are time based models that aim to account for all aspects of the faculty's workload by estimating the time required for each of these activities. However, these approaches are largely ad-hoc in nature and require a considerable amount of effort required to model each faculty's workload. They are also generally perceived as micromanaging by being prescriptive about the exact amount of time spent on each activity.

    \item \textbf{Points based}

          These are similar to actual-hours based models, but quantify the activities in terms of points instead of hours. The equivalence between points and hours is loose, which allows some leeway in the time spent on the respective activities. However, the lack of a clear correlation between the two is not well-received by the faculty due to lack of transparency. The points are also biased towards activities with higher budgetary allocations and returns.

          The STAR Model (Service, Teaching, Administration, Research) proposed by \cite{finlay1994management} was a key stepping stone in defining the non-teaching parts of a faculty's workload. It focused more on allocating research time as a function of the other three activities. However, the approaches towards calculating each of the constituents were primitive.

          \cite{rohan2017} proposed an approach of treating research and service workloads in directly relaxing the teaching workload expectations from a faculty. It also proposes novel techniques to include factors like course newness, class size and the type of teaching activity concerned. It also proposes a unique view, where the number of research staff supervised by a faculty is as the proxy for research workload. However, even though the teaching workload relaxation approach works great in practice, it fails to give a comprehensive view of the faculty's workload to drive organizational decisions

          \begin{equation}
              T_c + T_a + T_p + S_i + S_p + S_c+R = C
              \label{griffith_wam}
          \end{equation}

          Recently, \cite{griffith2020framework} goes much further in describing the constituents of Service Workload as \textbf{Professional}, and \textbf{Community} Service Workload (\(S_p, S_c\)), and the constituents of Teaching workload as \textbf{Classroom}, \textbf{Administrative} and \textbf{Preparation} Teaching Workload (\(T_c, T_a, T_p\)) as shown in \autoref{griffith_wam}. This provides much needed clarity into the nature of the work involved in each of these constituents. It also highlights key problems the academic community faces in each of the workload constituents. However, it omits the inclusion of faculty-specific deviations to the teaching workload and the variation in workload depending on the faculty type.

\end{enumerate}

\section{Workload Inequity as a systemic problem}
\cite{jensen2009overload, kenny2014effectiveness} identify that workload modelling and distribution fails to make an impact due to the lack of additional resources made available to address workload inequities. The lack of available expertise in key areas disproportionately overloads certain faculty with said expertise, essentially punishing faculty for being skilled.

\cite{vardi2009impacts, houston2006academic} clearly identify that even with the mandated use of WAMs in Australian universities, overloading of faculty is a widespread problem due to under-supply issues.\cite{kenny2014effectiveness} also notes that defining realistic time limits is a necessary step towards ensuring welfare of academic staff.

\section{Allocation Priorities}

\cite{harwood1975optimizing} identifies the three key stakeholders of workload allocation and their expectations from the teaching workload distribution. \cite{schniederjans1987goal, badri1998multi} further go into the role faculty preferences play in this distribution.

\begin{enumerate}
    \item Faculty\\
          — Faculty course preferences are important as they allow them to align their teaching with their areas of research interest\\
          — Schedule preferences also allow them to create windows in their schedule for focused research and administrative work that fits their work styles
    \item Student Body\\
          — Students expect a minimum variety of courses that allows them to customize their academic journey
    \item Administration\\
          — Maintain equitable distribution of work among the staff and lack of bias
\end{enumerate}

\section{Existing Allocation Algorithms}

\cite{schniederjans1987goal} originally proposed a goal-programming algorithm for solving a preference-aware course allocation algorithm. Due to its matrix optimization methodology, however, it fails to apply on large-scale universities due to its high time complexity. It also suffers from an inability to deal with allocation priorities.

\cite{rohan2017} opted for a greedy solution due to the massive complexity involved in expressing a many-to-many allocation with multiple levels of course-priority in a goal-programming solution. The greedy solution involved iterating through the courses and allocating the best faculty available. The solutions produced by this approach were observed to be low-quality, with a tendency to converge local minima. This was countered using an Iterated Greedy algorithm inspired by the \cite{ruiz2007simple}, which goes through multiple phases of random destruction and reconstruction of the solution to converge towards a better solution.

\cite{munkres1957algorithms} is the go-to algorithm for matrix-optimization problems. However, it fails to apply to many-to-many solutions with inter-dependent allocations. Recently, \cite{zhu2016solving} has proposed an approach of applying the Munkres algorithm to many-to-many allocations, with promising results. However, due to the varying levels of course-priority present in teaching allocation, its applicability remains to be evaluated.

\cite{dofadar2021hybrid} has recently applied a combination of the Local Repair Algorithm and the Modified Genetic Algorithm, which shows promising results in the absence of any priority list in the algorithm. However, due to the absence of priority-optimization, it cannot be applied to a faculty-preference-aware allocation system.

\section{Issues with current WAMs}

\begin{enumerate}

    \item \textbf{Inability to accurately quantify Research Workload}

          Most WAMs do not account for the research workload. The few models that quantify the research aim to do so using the research output. However, these approaches assume an equal amount of time required for different research work and, thus, fail to account for the varying amounts of time different research activities require. It's also dangerous to base teaching output on the number of publications, since it incentivizes quantity over quality. Also, the inability to account for unpublished research is problematic.

    \item \textbf{Teaching Centric}

          Teaching Centric Models fail to give a clear picture of the workload distribution by treating service and research workloads as concessions to teaching, and thus fail to provide organizational insights into the overall workload distribution. \cite{rohan2017}

    \item \textbf{Misalignment with organizational goals}

          A University has organizational expectations set for most faculty types in the faculty handbook \cite{griffith2020framework}. However, most models fail to align to these organizational goals and do not establish a baseline of what is expected out of the faculty.

    \item \textbf{Retrospective approach}

          Of the few models that manage to quantify the research workload, most are retrospective in nature and eventually align to a Garbage-In-Garbage-Out due to their dependence on research output. This ensures that the current year will only be able to reach a research output similar to the previous year due to a proportional amount of teaching work allocated in this year.

    \item \textbf{Lack of realistic limits}

          \cite{vardi2009impacts} identifies that a key failure of workload models is the inability to recognize limits of workload a faculty can handle. This is largely due to treating workload as a zero-sum game without the key recommendation that more faculty needs to be hired to maintain equity.
\end{enumerate}


\section{Summary}

Upon looking into the WAMs previously used, there are clear limitations mentioned above that need to be accounted for in the design of a new WAM. One key area is aligning organizational goals with the workload model to incentivize its use by the administration. As \cite{vardi2009impacts} points out, the WAM also needs to be abstracted enough that it can be clearly understood.

It is also clearly demonstrated by \cite{kenny2014effectiveness} that WAMs should not be treated as a policing system, but instead used proactively to frame workload and policy decisions. Thus, a clear need for automating teaching workload distribution is felt. The existing allocation algorithms are fairly limited in terms of scalability and comprehensiveness. There are new developments in areas of many-to-many combinatorial optimization that could be considered as a potential approach to teaching workload allocation.

Further work is required in defining the allocation priorities better to encapsulate the various needs of the three stakeholders in teaching allocation. On top of this, measures need to be taken to solve the systemic problem of workload inequity, by highlighting the gaps in supply and possible measures that can be taken to rectify said gaps. The allocation system also needs to incorporate practical limitations of workload distribution to counter faculty overload to an impractical level. This is critical, as otherwise it would adversely impact teaching effectiveness.
