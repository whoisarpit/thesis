\chapter{Introduction}

This chapter briefly introduces the problem of holistic teaching workload allocation in research-intensive universities. It also describes the structure of the report and the contents of its corresponding chapters.

\section{Background and Motivation}

Teaching workload allocation and scheduling are some of the most crucial activities in educational institutions of all sizes. Although done once every semester, this activity forms the backbone of the university environment and governs how much time faculty members can spend in individual areas of the workload like preparing for lectures, delivering lectures, spending time on research and service work. Thus, it has a drastic impact on the academic performance of students, the quality of teaching, and the research outputs of the university.

In addition, the teaching workload allocation process is also a key factor in determining the satisfaction of faculty members. A good allocation process ensures that faculty members are allocated courses that they are interested in teaching and that they are allocated a fair amount of workload. This is especially important in research-intensive universities, where faculty members are expected to spend a significant amount of time on research. A good allocation process also ensures that faculty members are not overburdened with workload, which can lead to burnout and dissatisfaction. Additionally, it recognizes the fact that faculty members have different workload patterns and equitably distributes the workload among them.

Historically, this problem has been difficult to automate due to the various conflicting factors that govern the allocations, the massive scale of the problem in large universities, and difficulties in accurately quantifying the workload of faculty members. If done poorly, the allocation can lead to faculty dissatisfaction and inefficient usage of the precious time of students and faculty members.

This is a problem that has typically been solved manually with the help of spreadsheets and other tools. However, as the number of faculty members and courses increases, this process becomes increasingly difficult to manage, resulting in weeks of planning and manual effort to allocate the workload. Even with this considerable effort, it leads to a nearly static allocation that is carried forward from previous years. This leads to a lack of variety in the teaching schedule of faculty members, and an ever-increasing workload for some faculty members, while others are underutilized. This also makes it difficult to make changes to the curriculum, as the allocation is not flexible enough to accommodate these changes.

Although there are multiple in-house techniques and existing solutions to solve this problem, this problem is under-explored, every university has its own unique requirements, and there is no one-size-fits-all solution. Additionally, there are problems with workload inequity and faculty dissatisfaction which, although studied as a separate problem, are not considered holistically in the automated teaching allocation processes.

In this research exercise, we aim to explore the problem of teaching workload allocation in research-intensive universities and propose a solution that can be used to allocate teaching workload fairly and equitably, while also taking into account the research and service workload of faculty members.

\section{Research Objectives}

The primary objective of this research is to develop a holistic teaching workload allocation system that can be used to allocate teaching workload fairly and equitably. This should account for the student feedback on the quality of teaching, the preferences of faculty members, management requirements, and the workload of faculty members in other areas like research and service.

In the process of developing this system, we also aim to develop a model to accurately quantify the workload of faculty members in non-teaching areas like research and service. This model should then be able to provide an accurate representation of the various workload constituents of faculty members, which can then be used to allocate teaching workload fairly and equitably.

\section{Organization of the Report}

In this chapter, we discussed the motivation behind the development of a holistic teaching workload allocation system. We also discussed the objectives that we aim to achieve through this research exercise.

\textbf{Chapter 2} offers an overview of existing research and solutions in the area of teaching workload allocation. This includes research into the problems with existing allocation systems, various methods of quantifying and modeling faculty workload, existing solutions to the problem of teaching workload allocation, and other algorithms that are applicable to allocation problems in general. We then discuss the gaps in existing research and how this research aims to fill those gaps.

\textbf{Chapter 3} describes a systematic approach for measuring the research and service workloads of a faculty. This involves identifying the challenges in quantifying the research workload, before defining a technique to quantify it. Then the chapter describes the process of quantifying service workload, before defining the teaching workload of a faculty as a function of the research and service workload. It also describes a workload model that can be used to provide transparency and insight into the workload of faculty members.

\textbf{Chapter 4} describes the process of quantifying the workload required in teaching a lecture. This involves dividing the lecture workload into various constituents, and then for each constituent, defining the impact of class size and faculty familiarity with the course. It then defines the workload of a lecture as a function of these factors. It also describes a lecture-splitting strategy to avoid major deviations in lecture workload. Then it quantifies the workload of a tutorial and lab, before finally implementing and analyzing the workload model using real-world data.

\textbf{Chapter 5} describes the process of allocating lectures to faculty members. To achieve this, it first defines what constitutes a good allocation. It then models the course allocation as a mathematical problem and describes the process of solving it using the Hungarian algorithm. Building on this, it incorporates workload limits to ensure equitable allocation of workload and also incorporates the management's priority of preferential allocations of earlier-year courses. It then describes techniques for resolving unallocated lectures. These involve pre-emptive allocation of lectures with an acute shortage of faculty, dynamic swapping to resolve unallocated lectures, and targeted overloading of faculty to resolve unallocated lectures. Finally, it describes an approach for equitable overloading of all faculty to resolve unallocated lectures. It then combines all these processes to describe the lecture allocation process. Finally, it describes the results of the allocation process and the impact of various parameters on the allocation process.

\textbf{Chapter 6} describes the process of allocating tutorials and labs to faculty members. It first describes the process of modeling the allocation problem, building upon the lecture allocation process using the Hungarian algorithm. It then describes the objectives of the tutorial allocation process and defines the cost function to ensure that the objectives of fragmentation, and equitable allocation of workload are met. It also describes the process of batch allocation of tutorials, which involves allocating tutorials in batches to ensure that the objectives of fragmentation are met. It also introduces a Selective Workload Relaxation technique to resolve unallocated tutorials. It then describes the process of allocating labs to faculty members, using the same process, followed by the results of the tutorial and lab allocation process, and the impact of various parameters on it.

\textbf{Chapter 7} concludes the report by summarizing the research and its findings. It also describes the future work that can be done to improve the allocation process and the research.
